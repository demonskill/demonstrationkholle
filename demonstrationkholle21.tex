\documentclass{article}
\renewcommand*\familydefault{\sfdefault}
\usepackage[utf8]{inputenc}
\usepackage[T1]{fontenc}
\setlength{\textwidth}{481pt}
\setlength{\textheight}{650pt}
\setlength{\headsep}{10pt}
\usepackage{amsfonts}
\usepackage[T1]{fontenc}
\usepackage{palatino}
\usepackage{calrsfs}
\usepackage{geometry}
\geometry{ left=3cm, top=2cm, bottom=2cm, right=2cm}
\usepackage{xcolor}
\usepackage{amsmath}
\usepackage{tikz,tkz-tab}
\usepackage{cancel}
\usepackage{pgfplots}
\usepackage{pstricks-add}
\usepackage{pst-eucl}
\usepackage{amssymb}
\usepackage{icomma}
\usepackage{listings}
\begin{document}
\title{Démonstration kholle 20}
\date{}
\maketitle
	\renewcommand{\thesection}{\Roman{section}}
	\setlength{\parindent}{1.5cm}
\section{Tout idéal de $\mathbb K [X]$ est de la forme $P \mathbb K [X]$}

\section{En admettant le théorème de D'Alembert-Gauss: description des irréductibles de $\mathbb C [X]$, de $\mathbb R[X]$}
\section{Factorisation de $X^n-1$ dans $\mathbb R [X]$ selon la parité de n}
\section{Enoncer sans démonstration les relations coefficients-racines, les formules $\sum_{k=1}^n x_k^2= \sigma_1^2-2 \sigma_2$ et $\sum_{k=1}^n \frac{1}{x_k}=\frac{\sigma_{n-1}}{\sigma_{n}}$. Exemple : somme des carrés, cubes et inverses des racines de $X^3-3X+1$}
\section{Existence et unicité de la forme irréductibles d'une fraction rationnelle.}
\section{Coefficients d'un põle simple (formule $\frac{A(\alpha)}{B'(\alpha)}$), décomposition de $\frac{1}{X^n-1}$ dans $\mathbb{C}(X)$}
\section{Décomposition en élément simples de $\frac{P'}{P}$ lorsque P est scindé}
\end{document}
