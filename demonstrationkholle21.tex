\documentclass{article}
\renewcommand*\familydefault{\sfdefault}
\usepackage[utf8]{inputenc}
\usepackage[T1]{fontenc}
\setlength{\textwidth}{481pt}
\setlength{\textheight}{650pt}
\setlength{\headsep}{10pt}
\usepackage{amsfonts}
\usepackage[T1]{fontenc}
\usepackage{palatino}
\usepackage{calrsfs}
\usepackage{geometry}
\geometry{ left=3cm, top=2cm, bottom=2cm, right=2cm}
\usepackage{xcolor}
\usepackage{amsmath}
\usepackage{tikz,tkz-tab}
\usepackage{cancel}
\usepackage{pgfplots}
\usepackage{pstricks-add}
\usepackage{pst-eucl}
\usepackage{amssymb}
\usepackage{icomma}
\usepackage{listings}
\begin{document}
\title{Démonstration kholle 20}
\date{}
\maketitle
	\renewcommand{\thesection}{\Roman{section}}
	\setlength{\parindent}{1.5cm}
\section{Tout idéal de $\mathbb K [X]$ est de la forme $P \mathbb K [X]$}
\textcolor{red}{Définition :} \\
Un idéal de $\mathbb{K}[X]$ est une partie de I de $\mathbb K [X]$ tel que : \\
\textcolor{red}{1)} I est un sous-groupe de $(\mathbb K[X],+)$ \\
\textcolor{red}{2)}$\forall P \in I, \forall Q \in \mathbb K[X], PQ \in I$ \\
\textcolor{green}{Propriété :} \\
Pour $P \in \mathbb{K} [X]$, $P \mathbb{K}[X]$ est un idéal de $\mathbb K[X]$ \\
\textcolor{red}{Démonstration :} \\
\textcolor{red}{1)} $P\mathbb{K}[X] \subset \mathbb{K}[X]$ et
$ \neq \emptyset$ car $0 \in P \mathbb{K}[X]$ \\
Soit $(A,B) \in (P \mathbb K [X])^2 : P |A$ et $P|B$ donc $P|(A+B)$ \\
\textcolor{red}{2)} Soit $(A,B) \in (P \mathbb K [X]) \times \mathbb K[X]$ : \\
$P|A$ donc $P|AB$ $AB \in P \mathbb K [X]$
\section{En admettant le théorème de D'Alembert-Gauss: description des irréductibles de $\mathbb C [X]$, de $\mathbb R[X]$}
\textcolor{green}{Propriété :} \\
\textcolor{green}{1)} Les polynômes irréductibles de $\mathbb C [X]$ sont ceux de degré 1 \\
\textcolor{green}{2)} Dans $\mathbb R [X]$ : \\
 \indent -les polynômes de degré 1 \\
 \indent -ceux de dgré 2 à discriminant strictement négatif \\
 \textcolor{red}{Démonstration :} \\
\textcolor{green}{1)} $\mathbb  K = \mathbb{R}$ ou $\mathbb{C}, P \in \mathbb K[X]$ de degré 1 : \\
D'une part, P n'est pas constant. \\
D'autre part, si P=QR avec Q et R non constant : $deg(P)=deg(Q)+deg(R) \geq 2$ contradiction dons P est irréductible. \\
$\mathbb{K}= \mathbb C$, P irréductible montrons que deg(P)=1 \\
P irréductible donc non constant. Par le théorème de D'Alembert-Gauss : $\exists z \in \mathbb C, P(z)=0$ \\
$\exists Q \in \mathbb C [X], P=(X-z)Q$ or P irréductible donc Q constant d'où $deg(P)=1$ \\
\textcolor{green}{2)} $\mathbb K= \mathbb R$, P irréductible dans $ \mathbb R [X]$ on a de  même : \\
$\exists z \in \mathbb C, P(z)=0$ \\
{\bf $1^{er}$ cas :} $z\in \mathbb R$, idem que dans $\mathbb C[X]$, $P=(X-z)Q$ avec Q constant $deg(P)=1$ \\
{\bf $2^{eme}$ cas :} $z \in \mathbb C \backslash \mathbb R $ comme $P \in \mathbb R [X]$, $\bar{z}$ aussi est racine de P \\
Comme $z \neq \bar{z}$ : \\
$\exists Q \in \mathbb C [X],P=(X-z)(X-\bar{z})Q$ \\
$P=\underbrace{(X^2-2*Re(z)X+|z|^2)}_{\in \mathbb K [X]}Q$ \\
On constate que $Q$ est le quotient de la division euclidienne de $P \in \mathbb{R}[X]$ par $(X^2-2*Re(z)X+|z|^2) \in \mathbb R [X]$ donc $Q \in \mathbb{R}[X]$ \\
Comme P est irréductible dans $\mathbb{R}[X]$ Q est constante $deg(P)=2$ et P n'a pas de racine réelle donc son discriminant est strictement négatif
\section{Factorisation de $X^n-1$ dans $\mathbb R [X]$ selon la parité de n}
\section{Enoncer sans démonstration les relations coefficients-racines, les formules $\sum_{k=1}^n x_k^2= \sigma_1^2-2 \sigma_2$ et $\sum_{k=1}^n \frac{1}{x_k}=\frac{\sigma_{n-1}}{\sigma_{n}}$. Exemple : somme des carrés, cubes et inverses des racines de $X^3-3X+1$}
\section{Existence et unicité de la forme irréductibles d'une fraction rationnelle.}
\section{Coefficients d'un põle simple (formule $\frac{A(\alpha)}{B'(\alpha)}$), décomposition de $\frac{1}{X^n-1}$ dans $\mathbb{C}(X)$}
\section{Décomposition en élément simples de $\frac{P'}{P}$ lorsque P est scindé}
\end{document}
