\documentclass{article}
\renewcommand*\familydefault{\sfdefault}
\usepackage[utf8]{inputenc}
\usepackage[T1]{fontenc}
\setlength{\textwidth}{481pt}
\setlength{\textheight}{650pt}
\setlength{\headsep}{10pt}
\usepackage{amsfonts}
\usepackage[T1]{fontenc}
\usepackage{palatino}
\usepackage{calrsfs}
\usepackage{geometry}
\geometry{ left=3cm, top=2cm, bottom=2cm, right=2cm}
\usepackage{xcolor}
\usepackage{amsmath}
\usepackage{tikz,tkz-tab}
\usepackage{cancel}
\usepackage{pgfplots}
\usepackage{pstricks-add}
\usepackage{pst-eucl}
\usepackage{amssymb}
\usepackage{icomma}
\usepackage{listings}
\begin{document}
\title{Démonstration kholle 20}
\date{}
\maketitle
	\renewcommand{\thesection}{\Roman{section}}
	\setlength{\parindent}{1.5cm}
\section{Tout idéal de $\mathbb K [X]$ est de la forme $P \mathbb K [X]$}
\textcolor{red}{Définition :} \\
Un idéal de $\mathbb{K}[X]$ est une partie de I de $\mathbb K [X]$ tel que : \\
\textcolor{red}{1)} I est un sous-groupe de $(\mathbb K[X],+)$ \\
\textcolor{red}{2)} $\forall P \in I, \forall Q \in \mathbb K[X], PQ \in I$ \\
\textcolor{green}{Propriété :} \\
Tout idéal de $\mathbb K [X]$ est de la forme $P \mathbb K[X]$ avec $P$ unique à association près \\
\textcolor{red}{Démonstration :} \\
{\bf Unicité} On sait que : \\
$P_1 \mathbb K [X]=P_2 \mathbb K [X] \Longleftrightarrow P_1$ et $P_2$ associé \\
{\bf Existence :} Soit $I$ idéal de $\mathbb K [X] $ : \\
Si $I=\lbrace 0 \rbrace$: $P=0$ convient \\
Sinon, posons $E= \lbrace deg(B):B \in I \backslash \lbrace 0 \rbrace \rbrace $ \\
Alors $E \subset \mathbb{N}$ : \\
$E \neq \emptyset$ car $I \neq \lbrace 0 \rbrace$ et $0 \in I$ \\
Ainsi E possède un minimum d : \\
Soit $B \in I \backslash \lbrace 0 \rbrace$, tel que $deg(B)=d$ \\
Montrons que $I=B \mathbb K [X]$ : \\
{\boldmath$B \mathbb K [X] \subset I$ }: \\
$B \in I$ donc: $ \forall Q \in \mathbb K[X], BQ \in I$ \\
{\boldmath$I \subset B \mathbb K [X]$} :\\
Soit $A \in I$ comme $B \neq 0$ : \\
$\exists ! (Q,R) \in (\mathbb K [X])^2, (A=BQ+R)$ et $deg(R) < deg(B)$ \\
$R=A-BQ$ donc $R \in I$ or $deg(R)<deg(B)=d$ donc comme $d=min(E)$, $R=0$ \\
$A=BQ \in B \mathbb K [X]$
\section{En admettant  le théorème de D'Alembert-Gauss: description des irréductibles de $\mathbb C [X]$, de $\mathbb R[X]$}
\textcolor{green}{Propriété :} \\
\textcolor{green}{1)} Les polynômes irréductibles de $\mathbb C [X]$ sont ceux de degré 1 \\
\textcolor{green}{2)} Dans $\mathbb R [X]$ : \\
 \indent -les polynômes de degré 1 \\
 \indent -ceux de degré 2 à discriminant strictement négatif \\
 \textcolor{red}{Démonstration :} \\
\textcolor{green}{1)} $\mathbb  K = \mathbb{R}$ ou $\mathbb{C}, P \in \mathbb K[X]$ de degré 1 : \\
D'une part, P n'est pas constant. \\
D'autre part, si P=QR avec Q et R non constant : \\
$deg(P)=deg(Q)+deg(R) \geq 2$ contradiction donc P est irréductible. \\
$\mathbb{K}= \mathbb C$, P irréductible montrons que deg(P)=1 \\
P irréductible donc non constant. \\
Par le théorème de D'Alembert-Gauss : $\exists z \in \mathbb C, P(z)=0$ \\
$\exists Q \in \mathbb C [X], P=(X-z)Q$ \\
Or P irréductible donc Q constant d'où $deg(P)=1$ \\
\textcolor{green}{2)} $\mathbb K= \mathbb R$, P irréductible dans $ \mathbb R [X]$ on a de  même : \\
$\exists z \in \mathbb C, P(z)=0$ \\
{\bf \boldmath $1^{er}$ cas :} $z\in \mathbb R$, idem que dans $\mathbb C[X]$, $P=(X-z)Q$ avec Q constant $deg(P)=1$ \\
{\bf \boldmath $2^{eme}$ cas :} $z \in \mathbb C \backslash \mathbb R $ comme $P \in \mathbb R [X]$, $\bar{z}$ aussi est racine de P \\
Comme $z \neq \bar{z}$ : \\
$\exists Q \in \mathbb C [X],P=(X-z)(X-\bar{z})Q$ \\
$P=\underbrace{(X^2-2*Re(z)X+|z|^2)}_{\in \mathbb R [X]}Q$ \\
On constate que $Q$ est le quotient de la division euclidienne de $P \in \mathbb{R}[X]$ par $(X^2-2*Re(z)X+|z|^2) \in \mathbb R [X]$ donc $Q \in \mathbb{R}[X]$ \\
Comme P est irréductible dans $\mathbb{R}[X]$ Q est constante $deg(P)=2$ et P n'a pas de racine réelle donc son discriminant est strictement négatif
\section{Factorisation de $X^n-1$ dans $\mathbb R [X]$ selon la parité de n}
Factorisons $X^n-1$ dans $\mathbb R [X]$ \\
{\bf \boldmath $1^{er}$ cas :} n paire , $n=2p$ avec $ p \in \mathbb N^*$ \\
$e^{\frac{2ik\pi}{n}}=e^{\frac{ik\pi}{p}}, 0 \leq k \leq 2p-1$ \\
$z_0=1$ et $z_p=-1$, les autres $\in \mathbb C \backslash \mathbb R$ \\
Pour $k \in [[1,p-1]]:$ $\bar{z}_k=e^{\frac{-ik\pi}{p}}=e^{\frac{i(2p-k)\pi}{p}}=z_{2p-k}$ avec $2p-k \in [[p+1,2p-1]]$ \\
$X^{2p}-1=(X-1)(X+1)\prod_{k=1}^{p-1}\underbrace{(X-z_k)(X-\bar{z}_k)}_{=X^2-2Re(z)X+|z|^2}$ \\
$X^{2p}-1=(X-1)(X+1) \prod_{k=1}^{p-1}(X^2-2cos(\frac{k\pi}{p})X+1)$ \\
{\bf \boldmath $2^{eme}$ cas :} $n=2p+1 (p \in \mathbb N^*)$ \\
$z_k=e^{\frac{2ik\pi}{2p+1}},0 \leq k \leq 2p $ \\
$z_0=1$, les autres sont dans $\mathbb C \backslash \mathbb R$: \\
pour $k \in [[1,p]], \bar{z}_k=e^{-\frac{2ik\pi}{2p+1}}=e^{\frac{2i(2p+1-k)\pi}{2p+1}}$ avec $2p+1-k \in [[p+1,2p]]$ \\
$X^{2p+1}-1=(X-1) \prod_{k=1}^p (X-z_k)(X-\bar{z}_k)$ \\
$X^{2p+1}-1=(X-1) \prod_{k=1}^p (X^2-2cos(\frac{2k\pi}{2p+1})X+1)$
\section{Enoncer sans démonstration les relations coefficients-racines, les formules $\sum_{k=1}^n x_k^2= \sigma_1^2-2 \sigma_2$ et $\sum_{k=1}^n \frac{1}{x_k}=\frac{\sigma_{n-1}}{\sigma_{n}}$. Exemple : somme des carrés, cubes et inverses des racines de $X^3-3X+1$}
\textcolor{green}{Propriété :} \\
soit $P \in \mathbb K [X]$ de degré $n \geq 1$ alors : \\
\textcolor{green}{1)} la somme des multiplicités de ses racines est $\leq n$ \\
\textcolor{green}{2)} égalité si et seulement si P scindé (toujours le cas quand $\mathbb K = \mathbb C$) \\
\textcolor{green}{Propriété :} \\
\textcolor{green}{1)} $\sum_{k=1}^n x_k^2= \sigma_1^2-2 \sigma_2$ \\ \\
\textcolor{green}{2)} $\sum_{k=1}^n \frac{1}{x_k}=\frac{\sigma_{n-1}}{\sigma_{n}}$ \\
\textcolor{red}{Démonstration :} \\
\textcolor{green}{1)} $(\sum_{k=1}^n x_k)^2= \sum_{k=1}^n x_k^2+ 2 \sum_{i<j}x_ix_j$ \\
\textcolor{green}{2)} Mise au même dénominateur. \\
\textcolor{blue}{Exemple :} \\
$P=X^3-3X+1=(X-a)(X-b)(X-c)$ \\
avec $(a,b,c) \in \mathbb C^3$ : \\
$\sigma_1=a+b+c=0$ \\
$\sigma_2=ab+bc+ac=-3$ \\
$\sigma_3=abc=-1$ \\
\textcolor{blue}{1)} $a^2+b^2+c^2= \sigma_1^2-2\sigma_2=6$ \\
\textcolor{blue}{2)} $\frac{1}{a}+ \frac{1}{b} + \frac{1}{c}= \frac{\sigma_2}{\sigma_3}=3$ \\
\textcolor{blue}{3)} Comme $(a,b,c)$ sont racines de $X^2-3X+1$ on a : \\
$a^3+b^3+c^3=(3a-1)+(3b-1)+(3c-1)$ \\
$a^3+b^3+c^3=3 \sigma_1 -3 =-3$
\section{Existence et unicité de la forme irréductibles d'une fraction rationnelle.}
\textcolor{green}{Propriété :} \\
Tout $F \in \mathbb K (X)$ possède un couple de représentants premier entre eux . \\
Il est unique à association près. \\
\textcolor{red}{Démonstration :}\\
{\bf Existence : }Soit $(A,B) \in (\mathbb K[X])^2$ avec $B \neq 0$ tel que $F=\frac{A}{B}$ \\
Soit $D=A \wedge B \neq 0$ car ($B \neq 0$) \\
$\exists (U,V) \in (\mathbb K [X])^2, A=DU$ et $B=DV$ $V \neq 0$ car $B \neq 0$ \\
on a $F= \frac{U}{V}$ et $U \wedge V=1$ \\
{\bf Unicité :} si $F=\frac{A_1}{B_1}=\frac{A_2}{B_2}$ \\
avec $(A_1,A_2,B_1,B_2) \in (\mathbb K [X])^4$ $B_1$ et $B_2 \neq 0$ \\
$A_1 \wedge B_1= A_2 \wedge B_2=1$ \\
On a : \\
$A_1B_2=A_2B_1$ ainsi $B_1|A_1B_2$ or $A_1 \wedge B_1=1$ donc $B_1|B_2$ de même $B_2|B_1$ :\\
$\exists \lambda \in \mathbb K^*,B_2= \lambda B_1$ puis : \\
$\lambda A_1 B_1=A_2B_1$ \\
or $B_1 \neq 0$ donc $A_2= \lambda A_1$ (intégrité de $\mathbb K [X]$)
\section{Coefficients d'un põle simple (formule $\frac{A(\alpha)}{B'(\alpha)}$), décomposition de $\frac{1}{X^n-1}$ dans $\mathbb{C}(X)$}
\textcolor{green}{Propriété :} \\
Si $\alpha$ est un pôle simple de $F= \frac{A}{B}$ avec $A \wedge B=1$. \\
Alors le coefficient de $\frac{1}{X-\alpha}$ dans la décomposition en élément simple est $\frac{A(\alpha)}{B'(\alpha)}$ \\
\textcolor{red}{Démonstration :} \\
Soit $Q \in \mathbb K [X]$ tel que $B=(X-\alpha)Q$ et $Q(\alpha) \neq 0$(en effet $A \wedge B =1$ donc $\alpha$ pôle simple signifie $\alpha$ racine simple de B) \\
Soit $ \lambda$ le coefficient cherché : \\
D'une part : $\lambda=(X-\alpha)F(X)|_{\alpha}= \frac{A(\alpha)}{Q(\alpha)}$  \\
D'autre part : $B'=Q+(X-\alpha)Q'$ donc $B'(\alpha)=Q(\alpha)$ \\
\textcolor{blue}{Exemple :} \\
$\frac{1}{X^n-1}(\mathbb K = \mathbb C, n \geq 1)$ \\
on a la factorisation : $X^n-1=\prod_{k=0}^{n-1}(X-\omega^k), \omega=e^{\frac{2i\pi}{n}}$ \\
Donc $\frac{1}{X^n-1}= \sum_{k=0}^{n-1} \frac{\lambda_k}{X-\omega^k}$ \\
Avec $A=1$ et $B=X^n-1$, $B'=nX^{n-1}$ \\
$\lambda_k=\frac{A(\omega^k)}{B'(\omega^k)}=\frac{1}{n(\omega^k)^{n-1}}=\frac{\omega^k}{n}$ car $(\omega^k)^n=1$ \\
On a donc : $\frac{1}{X^n-1}=\frac{1}{n}\sum^{n-1}_{k=0}\frac{\omega^k}{X-\omega^k}$
\section{Décomposition en élément simples de $\frac{P'}{P}$ lorsque P est scindé}
Soit P scindé :  \\
$P= \lambda \prod_{k=1}^r (X-\alpha)^{m_k}$ avec $\lambda \in \mathbb K^*,r \in \mathbb N^*,(\alpha_1,...,\alpha_r) \in \mathbb K^r$, racines distinctes de multiplicités $(m_1,....,m_r)$\\
 alors  $\frac{P'}{P}=\sum_{k=1}^r \frac{m_k}{X-\alpha_k}$ \\
\textcolor{red}{Démonstration :} \\
$P=\lambda P_1 ... P_r$ avec $P_k=(X-\alpha_k)^{m_k}$ et $P'=\lambda \sum_{k=1}^rP_1...P'_k...P_r$ \\
$\frac{P'}{P}=\sum_{k=1}^r \frac{P'_k}{P_k}=\sum^r_{k=1} \frac{m_k}{X-\alpha_k}$
\end{document}
