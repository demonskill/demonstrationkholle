\documentclass{article}
\renewcommand*\familydefault{\sfdefault}
\usepackage[utf8]{inputenc}
\usepackage[T1]{fontenc}
\setlength{\textwidth}{481pt}
\setlength{\textheight}{650pt}
\setlength{\headsep}{10pt}
\usepackage{amsfonts}
\usepackage[T1]{fontenc}
\usepackage{palatino}
\usepackage{calrsfs}
\usepackage{geometry}
\geometry{ left=3cm, top=2cm, bottom=2cm, right=2cm}
\usepackage{xcolor}
\usepackage{amsmath}
\usepackage{tikz,tkz-tab}
\usepackage{cancel}
\usepackage{pgfplots}
\usepackage{pstricks-add}
\usepackage{pst-eucl}
\usepackage{amssymb}
\usepackage{icomma}
\usepackage{listings}
\begin{document}
\title{Informatique TP 10 section 3}
\date{}
\author{Gatt Guillaume}
\maketitle
	\renewcommand{\thesection}{\Roman{section}}
	\setlength{\parindent}{1.5cm}
\lstset{language=Python}
\begin{lstlisting}
#Question 11 
def somme(P,Q) :
    if len(P)>len(Q) :
        k=len(P)-len(Q)
        L=[0 for i in range(k)]
        Q= Q +L[ : ]
    elif len(P)<len(Q) :
        k=len(Q)-len(P)
        L=[0 for i in range(k)]
        P= P +L[ : ]
    S=[]
    for i in range(len(P)) :
        S.append(P[i] +Q[i])
    return S
#Question 12
def mult_scal(P,a) :
    L=[]
    for i in range(len(P)) :
        L.append(a*P[i])
    return L
#Question 13
def prod(P,Q) :
    L=[0 for i in range(len(P)+len(Q)-1)]
    for i in range(len(P)) :
        for j in range(len(Q)) :
            L[i+j]+=P[i]*Q[j]
    return L
#Question 14
def Lagrange(A,B) :
    n=len(A)
    L=[0]
    for i in range(n):
        P=[B[i]]
        for j in range(n):
            if j!= i :
                P=prod(P,[-A[j]/(A[i]-A[j]),1/(A[i]-A[j])])
        L=somme(L,P)
    return L
#Question 15
def matrice_poly(A):
    M=[]
    for i in range(len(A)) :
        L=[]
        for j in range(len(A)):
            L.append(A[i]**j)
        M.append(L)
    return M
def Lagrange2(A,B) :
    M=matrice_poly(A)
    Y=[[B[i]] for i in range(len(A))]
    return gauss(M,Y)
\end{lstlisting}
{\bf Question 13 :} \\ 
La fonction {\bf prod(P,Q)} est de complexité avec n la taille de P et m la taille de Q $O(n \times m)$ \\ 
{\bf Question 14 :} \\ 
La fonction {\bf Lagrange(A,B)} a une complexité $O(n^2)$ avec n la taille de A \\ 
{\bf Question 15 :} \\ 
La matrice M associé est : \\ 
M=
$\begin{pmatrix}
1 & a_1 & {a_1}^2 &... & {a_1}^{n-1} \\
1 & a_2 & {a_2}^2 & ... & {a_2}^{n-1} \\
\ddots & \ddots &\ddots & \vdots &\vdots\\ 
1 & a_{n-1} & (a_{n-1})^2 & ...& ({a_{n-1}})^{n-1}\\
1 & a_n & {a_n}^2 & ... &{a_n}^{n-1} \\
\end{pmatrix}
$
\end{document}