\documentclass{article}
\renewcommand*\familydefault{\sfdefault}
\usepackage[utf8]{inputenc}
\usepackage[T1]{fontenc}
\setlength{\textwidth}{481pt}
\setlength{\textheight}{650pt}
\setlength{\headsep}{10pt}
\usepackage{amsfonts}
\usepackage[T1]{fontenc}
\usepackage{palatino}
\usepackage{calrsfs}
\usepackage{geometry}
\geometry{ left=3cm, top=2cm, bottom=2cm, right=2cm}
\usepackage{xcolor}
\usepackage{amsmath}
\usepackage{tikz,tkz-tab}
\usepackage{cancel}
\usepackage{pgfplots}
\usepackage{pstricks-add}
\usepackage{pst-eucl}
\usepackage{amssymb}
\usepackage{icomma}
\usepackage{listings}
\begin{document}
\title{Démonstration kholle 27}
\date{}
\maketitle
\renewcommand{\thesection}{\Roman{section}}
\setlength{\parindent}{1.5cm}
\section{Cardinal d'un produit, extension à $\bigcup_{x\in A} \lbrace x \rbrace B_x$ avec tous les $B_x$ de  même cardinal (principe des bergers)}
\textcolor{green}{Propriété :} \\
Si E et F sont finis $E \times F $ aussi et $Card(E \times F)= Card(E) Card(F)$ \\
\textcolor{red}{Démonstration :}
{\bf \boldmath Si E ou F $= \emptyset$} alors $E \times F= \emptyset$ $0=0$ \\
Si $E= \lbrace x_0 \rbrace$, singleton \\
Soit $f: F \rightarrow \lbrace x_0 \rbrace \times F $ \\
$y \maps (x_0,y)$
et $g : \lbrace x_0 \rbrace \times F \rightarrow F$ \\
$(x_0,y) \mapsto y$ \\
Ces applications sont trivialements réciproques. \\
ainsi : f est bijective $\lbrace x_0 \rbrace \times F$ est finie et $Card(\lbrace x_0 \rbrace \times F)= Card(F) $
Si $x_1,..., x_n$ sont les éléments distincts de E : \\
Notons $F_k= \lbrace x_k \rbrace $ et $F(1 \leq k \leq n)$ \\
Alors : $(F_1,...,F_n)$ est une partition de $E \times F$
\section{Dénombrement des arrangements, des injections}
\section{Expression des combinaisons avec des factorielles}
\section{Cardinal de $\mathcal P (E)$ et $\sum_{k=0}^n \binom{n}{k}=2^n$ }
\section{Formule du triangle de Pascal.}
\section{Démonstration combinatoire de $\binom{n}{p}= \frac n p \binom{n-1}{p-1}$}
\section{Exercice : calcul combinatoire de $\sum_{k=p}^n \binom{p-1}{k-1} $ }
\section{ Exercice : Formule de Vandermonde}
\end{document}
