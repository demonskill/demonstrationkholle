\documentclass{article}
\renewcommand*\familydefault{\sfdefault}
\usepackage[utf8]{inputenc}
\usepackage[T1]{fontenc}
\setlength{\textwidth}{481pt}
\setlength{\textheight}{650pt}
\setlength{\headsep}{10pt}
\usepackage{amsfonts}
\usepackage[T1]{fontenc}
\usepackage{palatino}
\usepackage{calrsfs}
\usepackage{geometry}
\geometry{ left=3cm, top=2cm, bottom=2cm, right=2cm}
\usepackage{xcolor}
\usepackage{amsmath}
\usepackage{tikz,tkz-tab}
\usepackage{cancel}
\usepackage{pgfplots}
\usepackage{pstricks-add}
\usepackage{pst-eucl}
\usepackage{amssymb}
\usepackage{icomma}
\usepackage{listings}
\begin{document}
\title{Démonstration kholle 27}
\date{}
\maketitle
\renewcommand{\thesection}{\Roman{section}}
\setlength{\parindent}{1.5cm}
\section{Cardinal d'un produit, extension à $\bigcup_{x\in A} \lbrace x \rbrace B_x$ avec tous les $B_x$ de  même cardinal (principe des bergers)}
\textcolor{green}{Propriété :} \\
Si E et F sont finis $E \times F $ aussi et $Card(E \times F)= Card(E) Card(F)$ \\
\textcolor{red}{Démonstration :}
{\bf \boldmath Si E ou F $= \emptyset$} alors $E \times F= \emptyset$ $0=0$ \\
Si $E= \lbrace x_0 \rbrace$, singleton \\
Soit $f: F \rightarrow \lbrace x_0 \rbrace \times F $ \\
$y \mapsto (x_0,y)$
et $g : \lbrace x_0 \rbrace \times F \rightarrow F$ \\
$(x_0,y) \mapsto y$ \\
Ces applications sont trivialements réciproques. \\
ainsi : f est bijective $\lbrace x_0 \rbrace \times F$ est finie et $Card(\lbrace x_0 \rbrace \times F)= Card(F) $
Si $x_1,..., x_n$ sont les éléments distincts de E : \\
Notons $F_k= \lbrace x_k \rbrace $ et $F(1 \leq k \leq n)$ \\
Alors : $(F_1,...,F_n)$ est une partition de $E \times F$
en effet :
$F_k \subset E \times F$ pour tout k donc $\bigcup_{k=1}^n F_k \subset E \times F$ \\
Soit $(x,y) \in E \times F$ \\
Alors : $\exists k \in [[1,n]],x=x_k$ \\
Or $E= \lbrace x_1,..,x_n \rbrace $ donc $(x,y) \in F_k$ \\
A fortiori  $E \times F \subset \bigcup_{k=1}^n F_k$ \\
si $i \neq j$ et $(x,y) \in F_i \cap F_j$ : \\
$x=x_i$ et $x=x_j$ \\
or $x_i\neq x_j $ \\
ainsi : $F_i \cap F_j = \emptyset $ \\
On a bien ; \\
$E \times F= \bigcup_{k=1}^n F_k$  avec les $F_k$ disjoint \\
$E \times F$ est fini et : \\
$Card(E\times F)= \sum_{k=1}^n Card(F_k)$ \\
Or on a vu prédemment : $\forall k \in [[1,n]], Card(F_k)=Card(F)$ \\
donc $Card(E \times F) = n Card(F)=Card(E) Card(F)$ \\
\textcolor{green}{Propriété :} \\
Si on effectue un $1^{er}$ choix parmi n options puis un second ayant p options (pouvant dépendre du $1^{er}$ choix), il y a np couples de choix possibles \\
Formellement : $A$ fini de cardinal $n \geq 1$ \\
F ensemble quelconque et, pour tout $x \in A$ \\
Soit $B_x \subset F$ fini de cardinal $p \in \mathbb N^*$ (indépendant de x) \\
Alors : \\
$Card ( \bigcup_{x \in A}\underbrace{(\lbrace x \rbrace \times B_x))}_{\lbrace (x,y),y\in B_x \rbrace}=np$ \\
\textcolor{red}{Démonstration :} \\
Soit $G= \bigcup_{x \in A} {(\lbrace x \rbrace \times B_x))}$ \\
l'union est disjointe car, si $x \neq x'$ : \\
$\lbrace x \rbrace \times B_x = \lbrace (x,y) : y\in B_x \rbrace $ \\
$ \lbrace x' \rbrace \times B_{x'}=\lbrace (x',z) : z \in B_{x'}$ \\
pas d'élément commun car la première composante diffère toujours \\
De plus : \\
$\forall x \in A, Card(\lbrace x \rbrace \times B_x)= \underbrace{Card(B_x)}_{p}$ \\
donc G est fini de cardinal $np$
\section{Dénombrement des arrangements, des injections}
\textcolor{green}{Propriété : } \\
pour $0 \leq p \leq n $, $A_n^p=\frac{n!}{(n-p)!}$ \\
\textcolor{red}{Démonstration :} \\
si $p=0: A_n^0=1$ (liste vide) \\
si $p \geq 1$ : on a n choix pour la première composante \\
puis n-1 choix pour la seconde composante \\
$\vdots$ \\
$\vdots$ \\
puis $n-p-1$ choix pour la p-ième composante \\
donc : $A_n^p=n(n-1)...(n-p+1)$ \\
$A_n^p=\frac{n!}{(n-p)!}$ \\
\textcolor{green}{Propriété :} \\
E fini de cardinal p et F fini de cardinal n alors $Card(Inj(E,F))=A_n^p$ \\
\textcolor{red}{Démonstration: } \\
Notons $x_1,...,x_p$ les éléments distincts on a vu que : \\
$\Phi : F^E \rightarrow F^p$ \\
$f \mapsto (f(x_1),...,f(x_p))$ est une bijection \\
ainsi : $\psi=\Phi_{|Inj(E,F)} $ \\
$\psi$ est une injection de $Inj(E,F)$ vers $F^p$, elle réalise donc une bijection sur son image ainsi : \\
$Card(Inj(E,F))=Card(Im(\psi))$ \\
Or $Im(\psi)$ est l'ensembles des p-arrangements de f : \\
$Card(Inj(E,F))=A_n^p$ 
\section{Expression des combinaisons avec des factorielles}
\section{Cardinal de $\mathcal P (E)$ et $\sum_{k=0}^n \binom{n}{k}=2^n$ }
\textcolor{green}{Propriété :} \\
Si E est fini, $\mathcal P(E)$ aussi et $Card(\mathcal P(E))=2^{Card(E)}$
\textcolor{red}{Démonstration :} \\
Posons $\Phi : \mathcal (E) \rightarrow \lbrace 0,1 \rbrace^E$ \\
$A \mapsto 1|_{A}$ \\
$\Psi : \lbrace 0,1 \rbrace^E \rightarrow \mathcal P(E)$ \\
$f \mapsto \lbrace x \in E : f(x)=1 \rbrace=f^{-1}(\lbrace 1 \rbrace)$ \\
Elles sont réciproques ainsi : \\
$ \mathcal P(E)$ est fini de cardinal : $Card(\lbrace 0,1 \rbrace^E)=2^{Card(E)} $ \\
\textcolor{green}{Propriété :} \\
$\forall n \in \mathbb N, \sum_{k=0}^n \binom{n}{k}=2^n$ \\
\textcolor{red}{Démonstration :} \\
Soit E de cardinal n : \\
$\mathcal P(E)= \bigcup_{k=0}^n \mathcal P_k(E)$ avec $\mathcal P_k(E)$ disjoints donc : \\
$Card(\mathcal P(E))=\sum_{k=0}^n Card(\mathcal P_k(E)$ \\
$2^n=\sum_{k=0}^n \binom{n}{k}$
\section{Formule du triangle de Pascal.}
\textcolor{green}{Propriété :} \\
pour $1 \leq p \leq n$ \\
$\binom{n}{p}=\binom{n-1}{p} +\binom{n-1}{p-1}$ \\
\textcolor{red}{Démonstration :}
$E=[[1,n]]$ \\
$\mathcal U= \lbrace A \in \mathcal P(E): Card(A)=p$ et $n\in A \rbrace$ \\
$\mathcal V= \lbrace A \in \mathcal P(E) : Card(A)=p-1$ et $n \notin A \rbrace$ \\
Alors : $\mathcal P(E)= \mathcal U \cup \mathcal V$ \\
donc $\binom{n}{p}= Card(\mathcal  U) + Card(\mathcal V)$ \\
on a $\mathcal V= \lbrace A \subset [[1,n-1]] : Card(A)=p \rbrace $ \\
donc $Card(\mathcal V)=\binom{n-1}{p}$ \\
Pour $\mathcal U$ posons : \\
$\phi : \mathcal U \rightarrow \mathcal P_{p-1}([[1,n-1]])$ \\
$A \mapsto A \backslash \lbrace n \rbrace$ \\
$\psi : \mathcal P_{p-1}([[1,n-1]])\rightarrow \mathcal U$ \\
$A \mapsto A \cup \lbrace n \rbrace $ \\
Ces applications sont réciproques ainsi : \\
$Card(U)= \binom{n-1}{p-1}$
\section{Démonstration combinatoire de $\binom{n}{p}= \frac n p \binom{n-1}{p-1}$}
\section{Exercice : calcul combinatoire de $\sum_{k=p}^n \binom{p-1}{k-1} $ }
\section{ Exercice : Formule de Vandermonde}
\end{document}
