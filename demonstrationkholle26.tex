\documentclass{article}
\renewcommand*\familydefault{\sfdefault}
\usepackage[utf8]{inputenc}
\usepackage[T1]{fontenc}
\setlength{\textwidth}{481pt}
\setlength{\textheight}{650pt}
\setlength{\headsep}{10pt}
\usepackage{amsfonts}
\usepackage[T1]{fontenc}
\usepackage{palatino}
\usepackage{calrsfs}
\usepackage{geometry}
\geometry{ left=3cm, top=2cm, bottom=2cm, right=2cm}
\usepackage{xcolor}
\usepackage{amsmath}
\usepackage{tikz,tkz-tab}
\usepackage{cancel}
\usepackage{pgfplots}
\usepackage{pstricks-add}
\usepackage{pst-eucl}
\usepackage{amssymb}
\usepackage{icomma}
\usepackage{listings}
\begin{document}
\title{Démonstration kholle 26}
\date{}
\maketitle
\renewcommand{\thesection}{\Roman{section}}
\setlength{\parindent}{1.5cm}
\section{Dévellopement par rapport à une ligne ou une colonne.}
\textcolor{green}{Propriété :} \\
$A \in M_n(\mathbb K) $ \\
$A_{ij}$, obtenue en supprimant la ligne i et la colonne j de A ($A_{ij} \in M_{n-1}(\mathbb K)$) \\
$\Delta_{ij}=det(A_{ij})$ mineur principal i-j \\
\textcolor{green}{1)} Pour $j \in [[1,n]]$ fixé : \\
$det(A)=\sum_{i=1}^n (-1)^{i+j} a_{ij} \Delta_{ij}$ développement par rapport à la j-ième colonne \\
\textcolor{green}{2)} Pour $i \in [[1,n]]$ fixé : \\
$det(A)= \sum_{j=1}^n (-1)^{i+j} a_{ij} \Delta_{ij}$ développement par rapport à la i-ème ligne \\
\textcolor{red}{Démonstration :} \\
\textcolor{green}{1)} $det(A)=\sum_{i=1}^n a_{ij}$
$\begin{vmatrix} &0 & \\ & \vdots &\\
  coeffA & 1 & coeffA \\
  & \vdots & \\
  & 0 &
\end{vmatrix}$
par linéarité sur la j-ième colonne \\
par permutation circulaire des colonnes j à n, on ammène $\begin{pmatrix} 0 \\ \vdots \\ 1 \\ \vdots \\ 0\end{pmatrix}$ en position n \\
  puis permutation circulaire des lignes i à n, on ammène la ligne i en position n \\
  $det(A)= \sum_{i=1}^n a_{ij} \underbrace{(-1)^{n-j} (-1)^{n-i}}_{=(-1)^{2n}(-1)^{i+j}} \underbrace{\begin{vmatrix} & 0 \\
    A_{ij} & \vdots \\
    & 0 \\
    ? & 1\end{vmatrix}}_{\Delta_{ij}}$ \\
    $det(A)=\sum_{i=1}^n (-1)^{i+j} a_{ij} \Delta_{ij}$ \\
    \textcolor{green}{2)} $det(A)= \sum_{j=1}^n a_{ij} \begin{vmatrix} coeff A \\ 0 & \cdots & 1 & \cdots \\ coeff A \end{vmatrix}$
    cycle sur les lignes puis les colonnes  : \\
    $det(A)=\sum_{j=1}^n a_{ij} (-1)^{n-i}(-1)^{n-j} \begin{vmatrix}A_{ij} & &  & ? \\ 0 & \cdots & 0 & 1\end{vmatrix}$
      
\section{Relation $A {}^tCom(A)={}^tCom(A)A=det(A)I_n$.}
      \textcolor{green}{Propriété :} \\
      $\forall A \in M_n(\mathbb K),  A {}^t Com(A)= {}^t Com(A)= det(A)I_n$ \\
      \textcolor{red}{Démonstration : } \\
      coeff i-j de $A{}^t Com(A)$ \\
      $\sum_{k=1}^n a_{ik} (-1)^{j+k} \Delta_{jk}$(cofacteur j-k car transposée) \\
      {\bf si \boldmath $i = j$ :} On reconnait le développement de det(A) par rapport à la ligne i : \\
      $\sum_{k=1}^n (-1)^{i+k}a_{ik}\Delta_{il}=det(A)$ \\
      {\bf si \boldmath $i \neq j$ :} Soit B ayant les mêmes coefficients  que A sauf la ligne j, que l'on remplace par la ligne i de A. \\
      Alors $det(B) = 0$ car deux lignes sont égales dévellopement de B par rapport à la ligne j : \\
      $0=\sum_{k=1}^n (-1)^{j+k} b_{jk} \Delta_{jk}$ \\
      Comme B a les mêmes coefficient que A en hors de la ligne j : \\
      $0=a_{ij}$ par définitions de B \\
      idem avec les colonnes : \\
      $\sum_{k=1}^n (-1)^{i+k} \Delta_{ki} a_{kj}$ \\
      {\bf \boldmath si $i\neq j$ :} on remplace la colonne i de A par sa colonne j
\section{Déterminant triangulaire par blocs.}
\textcolor{green}{Propriété :} \\
  Avec $A \in M_{r} (\mathbb K)$ et $B \in M_{n-r}(\mathbb K)$ : \\
  $\begin{vmatrix} A & \ast \\ 0 & B \end{vmatrix}=det(A) det(B)$ \\
  et $\begin{vmatrix} A & 0 \\ \ast & B \end{vmatrix}=det(A) det(B)$ \\
  \textcolor{red}{Démonstration :} \\
  $\begin{pmatrix} A & C \\ 0 & B \end{pmatrix}= \begin{pmatrix} I_r & 0 \\ 0 & B \end{pmatrix} \begin{pmatrix} I_r & C \\ 0 & I_{n-r} \end{pmatrix}\begin{pmatrix} A & 0 \\ 0 & I_{n-r} \end{pmatrix} $ \\
  donc $\begin{vmatrix} A & C \\ 0 & B \end{vmatrix}= \begin{vmatrix} I_r & 0 \\ 0 & B \end{vmatrix} \begin{vmatrix} I_r & C \\ 0 & I_{n-r} \end{vmatrix}\begin{vmatrix} A & 0 \\ 0 & I_{n-r} \end{vmatrix} $
  or $\begin{vmatrix} I_r & 0 \\ 0 & B \end{vmatrix}= \begin{vmatrix} I_{r-1} & 0 \\ 0  & B\end{vmatrix}$\\
    par développement par rapport à la première colonne on repète jusqu'à $det(B)$ \\
    de meme : $\begin{vmatrix} A & 0 \\ 0 & I_{n-r} \end{vmatrix}= (-1)^{n+n} \begin{vmatrix} A & 0 \\ 0 & I_{n-r-1} \end{vmatrix}$ \\
      par développement par rapport à la dernière colonne on repète jusqu'à $det(A)$ \\
      enfin: $\begin{vmatrix} I_r& C \\ 0 & I_{n-r} \end{vmatrix}=1$ car $\begin{pmatrix} I_r & C \\ 0 & I_{n-r} \end{pmatrix}= \begin{pmatrix} 1& & \ast \\  & \ddots \\ (0) & & 1 \end{pmatrix} \in \mathcal T^s_n(\mathbb K)$
\section{Déterminant de Vandermonde}
\section{Inégalité de Cauchy-Schwarz et cas d'égalité.}
\section{Inégalité triangulaire et cas d'égalité}
\section{Propriétés variées de la norme + théorème de Pythagore}
\section{Propriétés générales de l'orthogonal d'une partie}
\end{document}
