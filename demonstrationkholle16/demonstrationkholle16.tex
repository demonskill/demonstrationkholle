\documentclass{article}
\renewcommand*\familydefault{\sfdefault}
\usepackage[utf8]{inputenc}
\usepackage[T1]{fontenc}
\setlength{\textwidth}{481pt}
\setlength{\textheight}{650pt}
\setlength{\headsep}{10pt}
\usepackage{amsfonts}
\usepackage[T1]{fontenc}
\usepackage{palatino}
\usepackage{calrsfs}
\usepackage{geometry}
\geometry{ left=3cm, top=2cm, bottom=2cm, right=2cm}
\usepackage{xcolor}
\usepackage{amsmath}
\usepackage{tikz,tkz-tab}
\usepackage{cancel}
\usepackage{pgfplots}
\usepackage{pstricks-add}
\usepackage{pst-eucl}
\usepackage{amssymb}
\usepackage{icomma}
\usepackage{listings}
\begin{document}
\title{Démonstration kholle 16}
\date{}
\maketitle
	\renewcommand{\thesection}{\Roman{section}}
	\setlength{\parindent}{1.5cm}
\section{Théorème de la division euclidienne (cas $(a,b) \in \mathbb{N} \times \mathbb{N}^\ast$)}
\textcolor{green}{Théorème :} \\ 
soit $(a,b) \in \mathbb{Z}^2$ avec $b \neq 0$ : \\ 
$\exists ! (q,r) \in \mathbb{Z}^2,(a=bq+r$ et $0 \leq r <|b|$ ) \\ 
C'est la division euclidienne de a par b  avec $a$ le dividende, $q$ quotient, $r$ reste \\
\textcolor{red}{Démonstration :} \\  
{\bf Unicité} : si $(q,r)$ et $(q',r')$ conviennent : \\ 
$a=bq+r$ et $a=bq'+r'$, ainsi que $0\leq r < |b|$ et $0 \leq r' < |b|$ \\ 
on a : $bq+r=bq'+r'$ \\ 
$r-r'=b(q'-q)$ or : \\ 
$0 \leq r < |b|$ et $-|b| < -r' \leq 0$ \\ 
$-|b| < r-r' < |b|$ \\ 
Or si $q \neq q'$ : \\ 
$|q'-q| \geq 1$ et $|r-r'|=|b||q-q'|\geq |b|$ \\ 
incompatible avec $-|b| < r-r' < |b|$ donc $q=q'$ par conséquent : \\ 
$r-r'=b(q-q')=0$, $r=r'$ \\ 
{\bf Existence :} distinguons trois cas: \\ 
{\bf Cas 1 :} $a \in \mathbb{N}, b \in \mathbb{N}^\ast$ \\ 
Posons $E= \lbrace k \in \mathbb{N}, bk \leq a\rbrace$ \\ 
On a : $E \subset \mathbb{N}$, $E \neq \emptyset$ car $0 \in E $ \\ 
E est majorée par a : pour $k \in E$ on a : \\ 
$k \leq \underbrace{b}_{\in \mathbb{N}^\ast} k \leq a$ \\ 
Ainsi, E possède un maximum q : On a $q \in E$ donc $bq \leq a$ \\ 
On a $(q+1) \notin E$ car $q+1>max(E)$ \\ 
Comme $(q+1) \in \mathbb{N}$, on a donc $b(q+1)>a$ \\ 
ainsi : $bq \leq a < bq+b$ \\
Posons $r=a-bq \in \mathbb{Z}$ \\ 
$a=bq +r$ et $ 0 \leq r <b=|b|$ car $b \in \mathbb{N}^\ast$ \\ 
{\bf Cas 2 :} $a \leq 0$, $b>0$ : 
on applique le premier cas à $(-a,b)$, il existe $(q,r) \in \mathbb{Z}^2$ tel que : \\ 
$-a=bq+r$ et $0 \leq r <|b|$ alors: \\ 
$a=-bq-r$ si r=0 $(-q,0)$ convient \\ 
sinon : $b(-q-1)+(b-r)$ $0< b-r <b$ car $r>0$, $(-q-1,b-r)$ convient \\ 
{\bf Cas 3 :} a quelconque, b<0 \\ 
avec a et b : $\exists(q,r) \in \mathbb{Z}^2,a=(-b)q+r$ et $0 \leq r <-b$ \\ 
$a=b(-q)+r$ le couple $(-q,r)$ convient car $0 \leq r < |b|$\\
\section{Sous-groupes de $(\mathbb{Z},+)$}
\textcolor{green}{Théorème :} \\ 
\textcolor{green}{1)} Pour $n \in \mathbb{Z}$, $n\mathbb{Z}=\lbrace nk : k \in \mathbb{Z} \rbrace$ est un sous groupe de $(\mathbb{Z},+)$ \\ 
\textcolor{green}{2)} Tout sous-groupe de $( \mathbb{Z},+)$ est de la forme $n \mathbb{Z}$ avec $n \in \mathbb{N}$ unique \\ 
\textcolor{red}{Démonstration :} \\ 
\textcolor{green}{1)} Soit $n \in \mathbb{Z}$ \\ 
{\bf a)} $n\mathbb{Z} \subset \mathbb{Z}$ par définition \\ 
{\bf b)} $0=n \times 0 \in n \mathbb{Z}$ donc $n \mathbb{Z} \neq \emptyset$ \\ 
{\bf c)} Soit $(x,y) \in (n \mathbb{Z})$ : \\ 
$\exists (k,l) \in \mathbb{Z}^2,x=nk$ et $y=nl$ alors : \\ 
$x-y = n\underbrace{(k-l)}_{\in \mathbb{Z}}$ \\ 
$x-y \in n\mathbb{Z}$ \\ 
\textcolor{green}{2)} Soit H un sous-groupe de $(\mathbb{Z},+)$ \\ 
{\bf cas 1 :} $H= \lbrace 0 \rbrace,$ on prend $n=0$ \\ 
{\bf cas 2 : } $H \neq \lbrace 0 \rbrace$ comme $H \neq \emptyset$ \\ 
$\exists h \in H , h \neq 0$  H est stable par opposé : $-h \in H$ \\
Ainsi : $\underbrace{(h,-h)}_{h>0 \quad ou \quad -h>0} \in H^2$ donc :\\
$H \cap \mathbb{N}^\ast\neq \emptyset$ posons donc : \\ 
$n= min(H \cap \mathbb{N}^\ast)$ et montrons que $H=n\mathbb{Z}$ \\ 
\underline{$n\mathbb{Z} \subset H$} : soit $k \in \mathbb{Z}$ \\ 
{\bf Si $k=0$} $n0=0 \in H$ \\ 
{\bf Si $k \geq 0:$} $nk= \underbrace{n+..+n}_{k \quad termes}$ \\
Or $n \in H$ (car $n \in H \cap \mathbb{N}^\ast$) et H stable par + donc $nk \in H$ \\ 
{\bf Si} $k\leq-1$, $nk=(-n)(-k)$  \\ 
Or $-k \in \mathbb{N}^\ast$ donc $n(-k) \in H $ (propriété précédente). \\ 
Or H stable par opposé donc $nk \in H$ \\ 
\underline{$H \subset n \mathbb{Z}$} : soit $h \in H$ \\ 
On a $n \in H \cap \mathbb{N}^\ast \subset \mathbb{N}^\ast$ \\ 
Soit la division euclidienne de $h$ par $n$ : \\ 
$b=nq+r$, $q \in  \mathbb{Z}$ , $0 \leq r <n$ \\ 
On a $h \in H$ et $nq \in H$ car $n \mathbb{Z} \subset H$ \\ 
donc $r=k-nq \in H$ car $H$ sous-groupe de $(\mathbb{Z},+)$ \\ 
Si on avait $r>0$, on aurait : \\ 
$r \in H \cap \mathbb{N}^\ast$ or $r<n=min(H \cap \mathbb{N}^\ast)$ contradiction \\ 
donc r=0 : $h=nq \in n \mathbb{Z}$
\section{Existence du PGCD et relation de Bézout + lemme de Gauss}
\textcolor{green}{Théorème :} \\ 
Soit $(a,b) \in \mathbb{Z}^2$ \\ 
Il existe un unique $d \in \mathbb{N}$ tel que : \\ 
$d|a,d|b$ et pour tout $c \in \mathbb{Z}$ tel que $c|a$ et $c|b$ alors $c|d$ \\ 
C'est le pgcd de $a$ et $b$, noté pgcd(a,b) ou $a \wedge b$ de plus : \\ 
$\exists (u,v) \in \mathbb{Z}^2, au+bv=d$ (Relation de Bézout) \\ 
\textcolor{red}{Démonstration :} \\ 
{\bf Unicité :} si $d_1$ et $d_2$ conviennent on a : \\ 
$d_1 |a$ et $d_1 |b$ donc $d_1|d_2$ \\ 
De même, $d_2|d_1$ or $d_1$ et $d_2\geq 0$ donc $d_1 =d_2$ \\ 
{\bf Existence :} posons l'ensemble $a \mathbb{Z} +b\mathbb{Z}=\lbrace au+bv : (u,v) \in \mathbb{Z}\rbrace$ \\ 
Il s'agit d'un sous-groupe de $(\mathbb{Z},+)$. En effet : \\ 
\textcolor{green}{1)} $0=a\times 0+b \times 0 \in a \mathbb{Z} +b\mathbb{Z}$ donc  $a \mathbb{Z} +b\mathbb{Z}\neq \emptyset$ \\ 
\textcolor{green}{2)} Soit $(x_1,x_2) \in (a \mathbb{Z} +b\mathbb{Z})^2$ : \\ 
$\exists (u_1,v_1,u_2,v_2) \in \mathbb{Z}^4, x_1=au_1+bv_1$ et $x_2=au_2+bv_2$ \\ 
$x_1-x_2=a\underbrace{(u_1-u_2)}_{\in \mathbb{Z}}+\underbrace{b(v_1-v_2)}_{\in \mathbb{Z}}$ donc $x_1 -x_2 \in a \mathbb{Z} +b\mathbb{Z}$ \\ 
Ainsi, il existe $d \in \mathbb{N}$ tel que : \\ 
$a \mathbb{Z} +b\mathbb{Z}=d\mathbb{Z}$ \\ 
Vérifions que $d$ convient : \\ 
On a: $d=d_1 \in d \mathbb{Z}= a \mathbb{Z} +b\mathbb{Z}$ donc : \\ 
$\exists (u,v) \in \mathbb{Z}, d=au+bv$ \\ 
De plus : $a=a \times 1 +b \times 0 \in a \mathbb{Z} +b\mathbb{Z}=d \mathbb{Z}$  donc $d|a$\\
$b= a \times 0 + b \times 1 \in a \mathbb{Z} +b\mathbb{Z}=d \mathbb{Z}$ donc $d|b$ \\ 
Enfin, soit $c \in \mathbb{Z}$ tel que $c|a$ et $c|b$ \\  
Par combinaison linéaire : $c|(au+bv)$ et $c|d$ \\ 
\textcolor{green}{Théorème :} \\ 
Lemme de Gauss : \\ 
Soit $(a,b,c) \in \mathbb{Z}^3$ tel que $a|bc$ et $a \wedge b=1$ alors $a|c$. \\ 
\textcolor{red}{Démonstration :} \\ 
Soit $(u,v) \in \mathbb{Z}^2$ tel que $au+bv=1$ : \\ 
$a\times cu+bc\times v=c$ \\ 
Or : $a|a$(refléxivité) et $a|bc$(hypothèse) donc par combinaison linéaire: \\ 
$a|(a\times uc + bc \times v)$ c'est-à-dire $a|c$
\section{Algorithme d'Euclide (avec le lemme)}
\textcolor{green}{Lemme :} \\
Soit $(a,b,q,r) \in \mathbb{Z}^4$ tel que $a=bq+r$ alors $a \wedge b =b \wedge r$. \\ 
C'est en particulier vrai si on a une division euclidienne \\ 
\textcolor{red}{Démonstration :} \\ 
{\bf D'une part :} $(b\wedge r)|b$ et $(b \wedge r)|r$ \\ 
donc par combinaison linéaire : $(b \wedge r)|(bq+r)$ donc$(b \wedge r)|a$ \\
ainsi $b \wedge r$ divise a et b donc $(b\wedge r )|(a\wedge b)$ \\
{\bf D'autre part :} $r=b(-q)+ a$ \\ 
donc par le point précédent : $(b \wedge a)|(r \wedge b)$ et $(a \wedge b) | (b \wedge r)$ \\ 
Ainsi $a \wedge b$ et $b \wedge r$ sont associés or ils sont $\geq 0$ donc $a \wedge b =b \wedge r$ \\ 
\textcolor{green}{Algorithme :} 
\lstset{language=Python}
\begin{lstlisting}
def pgcd(a,b) :
    x,y = a,b 
    while y!=0 :
        x,y=y,x%y
    return x
\end{lstlisting}
\textcolor{red}{Démonstration :} Notons $x_k$ et $y_k$ les valeurs de x et y après k itérations \\ 
Invariant: \underline{$x_k \wedge y_k=a \wedge b$} \\ 
$k=0,x=a \quad y=b$ \\ 
Soit $k \geq 1$ tel que $x_{k-1} \wedge y_{k-1}=a \wedge b$ \\ 
Si on effectue une k-ième itérations :$y_{k-1} \neq 0$ \\ 
Le corps de la boucle vérifie : $x_k=y_{k-1}$\\ 
$y_k=$ reste de la division euclidienne de $x_{k-1}$ et $y_{k-1}$ \\ 
Donc par le lemme : $x_k \wedge y_k =x_{k-1} \wedge y_{k-1}=a \wedge b$ \\ 
{\bf Terminaison :} Si $b \geq 0$, les valeurs $y_k$ est une suite décroissante de $\mathbb{N}$ elle est donc finie. \\ 
Si $b<0$ à partir du rang 1  car reste $\geq 0$ \\ 
{\bf Correction :} notons $k_0$ le nombre d'itérations effectuées.\\ 
La condition de la boucle montre que $y_{k_0}=0$ \\
Mais $a \wedge b =x_{k_0} \wedge \underbrace{y_{k_0}}_{=0}$ (invariant) \\ 
$a \wedge b=x_{k_0}$ on renvoie bien $a \wedge b$ (au signe près) 
\section{$(a \wedge b)(a \vee b)= |ab|$}
\textcolor{green}{Propriété :} \\ 
$\forall (a,b) \in \mathbb{Z}^2, ( a \wedge b)(a \vee b)=|ab|$ \\ 
\textcolor{red}{Démonstration :} \\ 
Si $a=b=0$ : $a \wedge b=0$ \\ 
Si $(a,b) \neq (0,0)$ notons $d=a\wedge b \neq 0$ et $m=a \vee b$ \\ 
posons $\mu=\frac{ab}{d}$ \\ 
On a $\mu=a \times \underbrace{(\frac{b}{d})}_{\in \mathbb{Z}}\in \mathbb{Z}$ donc $a | \mu$\\
De même : $\mu=b\underbrace{(\frac{a}{d})}_{\in \mathbb{Z}}$ donc $b|\mu$ \\ 
Ainsi : $m| \mu$ \\ 
{\bf D'autre part :} \\ 
$\exists (u,v) \in \mathbb{Z}^2,au+bv=d$ puis : \\ 
$amu+bmv=dm$ \\ 
Or : $b |m$ donc $ab|am$ et $a|m$ donc $ab|bm$ \\ 
donc par combinaison linéaire : \\ 
$ab|(amu+bmv)$ c'est-à-dire $ab|dm$ \\ 
donc : $d\mu |dm$ enfin $d \neq 0$ donc $\mu |m$ \\ 
Conclusion : $|\mu|=|m|$ donc $dm=|ab|$
\section{Tout entier $\geq 2$ possède un diviseur premier + infinité de l'ensemble des nombres premiers}
\textcolor{green}{Propriété :} \\ 
Tout entier $\geq 2$ possède un diviseur premier. \\ 
\textcolor{red}{Démonstration :} Soit $n \geq 2$ \\ 
Soit $D=\lbrace d \in \mathbb{N}^\ast : d \geq 2$ et $d|n \rbrace$ \\ 
On a $D \subset \mathbb{N}$ par définition et $n \in D \neq \emptyset$ \\ 
Posons donc $p=min(D)$ : \\ 
On a $p \in D$ donc $p|n$ mais aussi $p \geq 2$. \\ 
Soit $d \in \mathbb{N}^ \ast$ tel que $d|p$ : \\ 
Comme $p|n$, on a donc $d|n$ si $d \neq 1$, on a donc : $d \in D$ donc $d \geq p$ \\ 
Or $d|p$, d'où $d=p$ \\ 
Ainsi $p\geq 2$ est divisible seulement par $1$ et $p$ donc p est premier. \\
\textcolor{green}{Propriété :} \\ 
L'ensemble des nombres premier est infini. \\ 
\textcolor{red}{Démonstration :} \\ 
S'il existaient qu'un nombre fini de nombres premiers $p_1,...,p_r$ : \\ 
Prenons $N=p_1..p_r+1 \geq 2$ \\ 
On a montré que N possédait un diviseur premier : \\ 
$\exists i \in [[1,r]], p_i |N$ \\ 
Or $p_i|p_1...p_r$ donc par combinaison linéaire : \\ 
$p_i|(N-p_1...p_r)$ donc $p_i|1$ \\ 
Absurde car $p_i \geq 1$
\section{Tout entier $\geq 2$ possède un diviseur premier + existence de la décomposition en facteurs premiers}
\textcolor{green}{Propriété :} \\ 
Tout entier $\geq 2$ possède un diviseur premier. \\ 
\textcolor{red}{Démonstration :} Soit $n \geq 2$ \\ 
Soit $D=\lbrace d \in \mathbb{N}^\ast : d \geq 2$ et $d|n\rbrace$ \\ 
On a $D \subset \mathbb{N}$ par définition et $n \in D \neq \emptyset$ \\ 
Posons donc $p=min(D)$ : \\ 
On a $p \in D$ donc $p|n$ mais aussi $p \geq 2$. \\ 
Soit $d \in \mathbb{N}^ \ast$ tel que $d|p$ : \\ 
Comme $p|n$, on a donc $d|n$ si $d \neq 1$, on a donc : $d \in D$ donc $d \geq p$ \\ 
Or $d|p$, d'où $d=p$ \\ 
Ainsi $p\geq 2$ est divisible seulement par $1$ et $p$ donc p est premier. \\
\textcolor{green}{Théorème :}\\ 
Tout entier $\geq 1$ s'écrit de façon unique (à l'ordre  des facteurs près) comme produit de nombres premiers. \\ 
\textcolor{red}{Démonstration :} \\ 
{\bf Existence :} Récurrence forte \\ 
\underline{Initialisation :} $n=1$, produit nulle \\ 
\underline{Hérédité :} soit $n \geq 2$ tel que tout entier $\geq 1$ et $<n$ soit produit de nombres de premiers : \\ 
soit p premier tel que $p|n$ soit $k=\frac{n}{p}\in \mathbb{N}$ : \\
$1 \leq k <n$ \\ 
donc k est produit de nombres premiers donc $n=p \times k$ l'est aussi \\ 
{\bf Unicité :} Récurrence forte \\ 
\underline{Initialisation :} $n=1$ seul le produit vide convient \\ 
\underline{Hérédité :} soit $n \geq 2$ tel que l'unicité soit vraie pour tout k avec $1 \leq k <n$ \\ 
Supposons : $n=p_1...p_r$ et $n=q_1..q_s$ avec $(r,q) \in \mathbb{N}^{\ast^2}$ \\ 
$p_i,q_j$ premiers et $p_1 \leq ... \leq p_r$ ainsi que $q_1 \leq ... \leq q_s$ \\ 
L'ensemble des diviseurs premiers de n est une partie non vide (car $n \geq 2$). \\ 
Il possède donc un minimum p : \\ 
On a $p|n$, c'est-à-dire $p|p_1...p_r$ et p premier donc (lemme de Gauss) : \\ 
$\exists i \in [[1,r]],p|p_i$ \\ 
Or p et $p_i$ premier donc $p=p_i$ \\ 
De plus : $p_1 \leq ... \leq p_i$ \\ 
D'où comme $p=p_i$ : \\ 
$p=p_1=...=p_i$ \\ 
De même $p=q_1$ \\ 
Posons $k= \frac{n}{p}=p_2...p_r=q_2...q_s$ \\ 
On a $1 \leq k < n$ donc par hypothèse de récurrence : \\ 
$r=s$ et $p_2=q_2,...,p_r=q_r$
\section{Petit théorème de Fermat (avec le lemme )}
\textcolor{green}{Lemme :} \\ 
Soit p premier et $k\in [[1,p-1]]$ alors $p|\binom{p}{k}$ \\ 
\textcolor{red}{Démonstration :} \\ 
$p\binom{p-1}{k-1}=k\binom{p}{k}$ \\ 
Comme p est premier on a par le lemme d'Euclide \\ 
$p|k$ ou $p|\binom{p}{k}$ \\ 
Les deux premiers multiples $\geq 0$ de p sont 0 et p or $0<k<p$ donc p ne divise pas k \\ 
Conclusion : $p|\binom{p}{k}$ \\ 
\textcolor{green}{Théorème :} \\ 
Soit p un nombre premier : \\ 
\textcolor{green}{1)} $\forall n \in \mathbb{Z}, n^p \equiv n [p]$ \\ 
\textcolor{green}{2)} Pour $n \in \mathbb{Z}$ non multiple de p : \\ 
$n^{p+1}\equiv 1[p]$ \\ 
\textcolor{red}{Démonstration :} \\ 
\textcolor{green}{1)} {\bf a)}Pour $n \in \mathbb{N}$ récurrence \\ 
\underline{Initialisation :} $n=0$ $0^p\equiv 0 [p]$ \\ 
\underline{Hérédité :} Soit $n \in \mathbb{N}$ tel que $n^p\equiv n[p]$ \\ 
alors : $(n+1)^p=\sum_{k=0}^p\binom{p}{k}n^k$ \\ 
$(n+1)^p\equiv n^p+1[p]$ car pour $0<k<p$ \\ 
$\binom{p}{k}n^k\equiv 0[p]$ par le lemme \\ 
Donc par hypothèse de récurrence : \\ 
$(n+1)^p \equiv (n+1)[p]$ \\ 
{\bf b)}Soit $n \in \mathbb{N}^\ast$ et montrons que $(-n)^p \equiv -n[p]$ \\ 
On a : $(-n)^p=(-1)^pn^p$ \\ 
$(-n)^p=-(n^p)$ si p impaire, c'est-à-dire $p \geq 3$ \\ 
$(-n)^p\equiv -n[p]$ par {\bf a} \\ 
Si $p=2$ : \\ 
$(-n)=n^2$ \\ 
$(-n)\equiv n[2]$ \\ 
$(-n) \equiv -n[2]$ car $-1 \equiv 1[2]$ \\ 
\textcolor{green}{2)} On a $n^p \equiv n[p]$ et de plus $non(p|n)$ \\ 
Comme p est premier : $p \wedge n=1$. Or p divise $n^p-n=n(n^{p-1}-1)$ \\ 
donc par le lemme de Gauss : $p|(n^{p-1}-1)$ \\ 
c'est-à-dire $n^{p-1}\equiv 1[p]$
\end{document}