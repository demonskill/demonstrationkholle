\documentclass{article}
\renewcommand*\familydefault{\sfdefault}
\usepackage[utf8]{inputenc}
\usepackage[T1]{fontenc}
\setlength{\textwidth}{481pt}
\setlength{\textheight}{650pt}
\setlength{\headsep}{10pt}
\usepackage{amsfonts}
\usepackage[T1]{fontenc}
\usepackage{palatino}
\usepackage{calrsfs}
\usepackage{geometry}
\geometry{ left=3cm, top=2cm, bottom=2cm, right=2cm}
\usepackage{xcolor}
\usepackage{amsmath}
\usepackage{tikz,tkz-tab}
\usepackage{cancel}
\usepackage{pgfplots}
\usepackage{pstricks-add}
\usepackage{pst-eucl}
\usepackage{amssymb}
\usepackage{icomma}
\usepackage{listings}
\begin{document}
\title{Démonstration kholle 20}
\date{}
\maketitle
	\renewcommand{\thesection}{\Roman{section}}
	\setlength{\parindent}{1.5cm}
\section{Division euclidienne}
\section{Factorisation d'une racine; un polynôme n'a pas plus de racines distincts que son degré}
\textcolor{green}{Propriété :} \\ 
Soit $P \in \mathbb{K}[X], \alpha \in \mathbb K$ \\
\textcolor{green}{1)} Le reste de la division euclidienne de P par $X-\alpha$ est $P(\alpha)$ \\ 
\textcolor{green}{2)} $P(\alpha)= 0 \Longleftrightarrow (X- \alpha)|P$ \\ 
\textcolor{red}{Démonstration :} \\ 
\textcolor{green}{1)} $P=(X-\alpha)Q + R$ \\ 
Avec $Q \in \mathbb K [X]$, $R \in \mathbb K [X]$ \\ 
et $deg(R)< deg(X- \alpha) = 1$ donc R est constant \\ 
substituons $\alpha$ à X : \\ 
$P(\alpha)=R(\alpha)=R$ car R constante \\ 
\textcolor{green}{2)} $(X- \alpha)|P \Longleftrightarrow R=0$ \\ 
\textcolor{green}{Propriété :} \\
	Soit $P \in \mathbb{K}[X]$ de degré $n \in \mathbb{N}$. Alors P possède au plus n racines. \\
	\textcolor{red}{Démonstration} : récurrence sur n \\
	\textcolor{red}{Initialisation} : n=0, $P \in \mathbb{K}^*$ aucune racine . \\
	\textcolor{red}{Hérédité :} Soit $ n \in \mathbb{N}^*$ tel que la propriété est vraie au rang n-1 \\
	Soit $ P \in \mathbb{K}[X]$ de degré n : \\
	-si P n'a pas de racine : il n'en a pas plus de n\\
	- si P a au moins une racine $c \in \mathbb{K}$,
		soit $Q \in \mathbb{K}[X]$ tel que : \\
		$\forall x \in \mathbb{R}, P(x)=(x-c)Q(x)$ \\
		alors $deg(Q)=n-1$ \\
			Q a donc au plus n-1 racines \\
			On a pour $x \in \mathbb{K}$ : \\
				$P(x)=0 \Leftrightarrow (x=c)$ ou $Q(x)=0$ \\
			P possède au plus une racine de plus que Q donc au plus n au total
\section{Factorisation simultanée de plusieurs racines distinctes ; cas où il y a autant de racines que le degré (polynôme scindé simple)}
\section{Interpolation de Lagrange}
\section{Dérivée de PQ}
\section{Formule de Taylor}
\section{Caractérisation différentielle de la multiplicité}
\end{document}