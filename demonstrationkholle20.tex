\documentclass{article}
\renewcommand*\familydefault{\sfdefault}
\usepackage[utf8]{inputenc}
\usepackage[T1]{fontenc}
\setlength{\textwidth}{481pt}
\setlength{\textheight}{650pt}
\setlength{\headsep}{10pt}
\usepackage{amsfonts}
\usepackage[T1]{fontenc}
\usepackage{palatino}
\usepackage{calrsfs}
\usepackage{geometry}
\geometry{ left=3cm, top=2cm, bottom=2cm, right=2cm}
\usepackage{xcolor}
\usepackage{amsmath}
\usepackage{tikz,tkz-tab}
\usepackage{cancel}
\usepackage{pgfplots}
\usepackage{pstricks-add}
\usepackage{pst-eucl}
\usepackage{amssymb}
\usepackage{icomma}
\usepackage{listings}
\begin{document}
\title{Démonstration kholle 20}
\date{}
\maketitle
	\renewcommand{\thesection}{\Roman{section}}
	\setlength{\parindent}{1.5cm}
\section{Division euclidienne}
\textcolor{green}{Propriété : } \\ 
Soit $(A,B) \in (\mathbb K [X])^2$ avec $B \neq 0$ \\ 
$\exists ! (Q,R) \in (\mathbb K [X])^2, (A=BQ+R)$ et $deg(R) < deg(B)$ \\ 
\textcolor{red}{Démonstration :} \\ 
{\bf Unicité :} \\ 
Si $A = BQ_1+R_1$ et $A=BQ_2 +R_2$ \\ 
avec $deg(R_1)$ e $deg(R_2)< deg (B)$ \\
$B(Q_1-Q_2)=R_2-R_1$ \\ 
$deg(B) +deg(Q_1-Q_2)=deg(R_2-R_1) $ \\ 
donc $deg(B)+deg(Q_1-Q_2)<deg(B)$ \\
or $deg(B) \in \mathbb N$ car $B  \neq 0$ $deg(Q_1-Q_2)<0$ \\ 
donc $Q_1-Q_2=0$ donc $Q_1=Q_2$ et $R_1=R_2$ \\ 
{\bf Existence :} \\ 
$Q,R \leftarrow 0,A$ \\ 
Tant que $deg(R) \geq deg(B)$ : \\ 
\indent Soit $sX^d$ le monome tel que $sX^dB$ ont le même terme de plus haut degré que R : \\
\indent $Q,R \leftarrow Q+sX^d,R-sX^dB$ \\ 
Renvoyer Q,R \\ 
Notons $Q_k$ et $R_k$ les valeurs de Q et R après k itérations \\
{\bf Terminaison :} $deg(R_k)$ décroit strictement dans $\mathbb N \cup \lbrace - \infty \rbrace$ \\ 
{\bf Correction} Montrons que qu'après chaque itérations on a bien $A=BQ_k+R_k$ \\ 
{\bf Initialisation :} $k=0, Q_0=0, R_0=A$ \\ 
$A=B \times 0 +A$ \\ 
{\bf Hérédité} Soit $ k \in \mathbb N^*$ tel que la propriété soit vraie au rang k-1. Si on effectue une k-ième itération : \\ 
$deg(R_{k-1}) \geq deg(B)$ \\ 
prenons $d=deg(R_{k-1}) -deg(B) \in \mathbb N$ et $s= \frac{dom(R{k-1})}{dom(B)}$ a un sens car $dom(B) \in \mathbb K ^*$ \\ 
Alors : $Q_k =Q_{k-1} + s X^d$ et $R_k=R_{k-1} -s X^dB$ \\ 
donc $B Q_k +R_k =B Q_{k-1} +R_{k-1} =A$ \\ 
{\bf Conclusion :} Soit $k_0$ le nombre total d'itérations, on renvoie $(Q_{k_0},R_{k_0})$ \\ 
Par principe de réccurrence : $A=BQ_{k_0} + R_{k_0}$ \\ 
Puisque l'on effectue plus d'itération supplémentaire $deg(R_{k_0}) < deg(B)$
\section{Factorisation d'une racine; un polynôme n'a pas plus de racines distincts que son degré}
\textcolor{green}{Propriété :} \\ 
Soit $P \in \mathbb{K}[X], \alpha \in \mathbb K$ \\
\textcolor{green}{1)} Le reste de la division euclidienne de P par $X-\alpha$ est $P(\alpha)$ \\ 
\textcolor{green}{2)} $P(\alpha)= 0 \Longleftrightarrow (X- \alpha)|P$ \\ 
\textcolor{red}{Démonstration :} \\ 
\textcolor{green}{1)} $P=(X-\alpha)Q + R$ \\ 
Avec $Q \in \mathbb K [X]$, $R \in \mathbb K [X]$ \\ 
et $deg(R)< deg(X- \alpha) = 1$ donc R est constant \\ 
substituons $\alpha$ à X : \\ 
$P(\alpha)=R(\alpha)=R$ car R constante \\ 
\textcolor{green}{2)} $(X- \alpha)|P \Longleftrightarrow R=0$ \\ 
\textcolor{green}{Propriété :} \\
	Soit $P \in \mathbb{K}[X]$ de degré $n \in \mathbb{N}$. Alors P possède au plus n racines. \\
	\textcolor{red}{Démonstration} : récurrence sur n \\
	\textcolor{red}{Initialisation} : n=0, $P \in \mathbb{K}^*$ aucune racine . \\
	\textcolor{red}{Hérédité :} Soit $ n \in \mathbb{N}^*$ tel que la propriété est vraie au rang n-1 \\
	Soit $ P \in \mathbb{K}[X]$ de degré n : \\
	-si P n'a pas de racine : il n'en a pas plus de n\\
	- si P a au moins une racine $c \in \mathbb{K}$,
		soit $Q \in \mathbb{K}[X]$ tel que : \\
		$\forall x \in \mathbb{R}, P(x)=(x-c)Q(x)$ \\
		alors $deg(Q)=n-1$ \\
			Q a donc au plus n-1 racines \\
			On a pour $x \in \mathbb{K}$ : \\
				$P(x)=0 \Leftrightarrow (x=c)$ ou $Q(x)=0$ \\
			P possède au plus une racine de plus que Q donc au plus n au total
\section{Factorisation simultanée de plusieurs racines distinctes ; cas où il y a autant de racines que le degré (polynôme scindé simple)}
\textcolor{green}{Propriété:} \\ 
Soit $P \in \mathbb K [X]$ \\
Soit $(\alpha_1,...,\alpha_r)$ des racines deux à deux distinctes de P. Alors : \\ 
$(X-\alpha_1)...(X-\alpha_r)|P$ \\ 
c'est-à-dire: $\exists Q \in \mathbb K  [X], P=(X-\alpha_1)...(X_\alpha_r)Q$\\ 
\textcolor{red}{Démonstration :} soit la division euclidienne : \\
$P=(X-\alpha_1)...(X_\alpha_r)Q +R$  avec $(Q,R) \in \mathbb K [X]$ \\
et $deg(R)<deg((X-\alpha_1) ...(X- \alpha_r)=r$ \\
Pour $k \in [[1,r]]$ : \\
$R(\alpha_k)=P(\alpha_k)=0$ donc R possède au moins $r$ racines or $deg(R)<r$ donc $R=0$\\
\textcolor{green}{Corrolaire :} \\
Soit $P \in \mathbb K [X]$ de degré $d \in \mathbb N^*$. \\
On suppose qu'il possède d racines distinctes 2à 2 $\alpha_1,...,\alpha_d$ alors : \\
$P=dom(P) \prod_{k=1}^d (X-\alpha_k)$
\textcolor{red}{Démonstration : } Comme les racines sont 2 à 2 disctincts : \\
$\exists Q \in \mathbb K [X], P=(X- \alpha_1)...(X- \alpha_d)Q$ \\ 
de degré $d=\underbrace{1+ ...+1}_{d \quad termes} + deg(Q)$ $deg(Q)=0$ donc $Q=\lambda \in \mathbb K^*$ \\ 
coefficient dominant : $dom(P)=\lambda$
\section{Interpolation de Lagrange}
\textcolor{green}{Propriété :} \\ 
Soit $n \in \mathbb N^*$, et $(a_1,...,a_n) \in \mathbb K^n$ deux à deux distincts. \\ 
La famille des polynomes associée à $(a_1,...,a_n=$ est $(L_1,...,L_n)$ où : \\ 
$L_k=\prod_{i=1,i \neq k}ˆn \frac{X-a_i}{a_k-a_i}$ \\ 
\textcolor{green}{1)} $\forall k \in [[1,n]], deg(L_k)=n-1$ \\ 
\textcolor{green}{2)} $\forall (k,j) \in [[1,n]]^2, L_k(a_j)=\delta_{k,j}$ \\ 
\textcolor{green}{3)} $(L_1,...,L_n)$ est une base de $\mathbb K_{n-1} [X] $ Plus précisement : \\ 
$\forall P \in \mathbb K_{n-1} [X], P= \sum_{k=1}^n P(a_k) L_k$ \\ 
\textcolor{red}{Démonstration :} \\ 
\textcolor{green}{1)} Il y a n-1 facteur de degré 1 \\ 
\textcolor{green}{2)} $L_k(a_j)=\prod_{i=1,i \neq k}ˆn \frac{a_j-a_i}{a_k-a_i}$ \\ 
si $j \neq k$, i peut prendre la valeur j et donne un facteur nul $L_k(a_j)=0$ \\ 
si $j=k$ : tous les facteurs valent 1 $L_k(a_j)=1$ \\ 
\textcolor{green}{3)} C'est bien une famille de $\mathbb K_{n-1} [X] $ \\ 
Montrons que tout $P \in \mathbb K_{n-1} [X]$ est combinaison linéaire de $(L_1,...,L_n)$ de façon unique \\ 
Soit $P \in \mathbb K_{n-1} [X]$ \\ 
{\bf analyse :} si $P= \sum_{k=1}^n \lambda_k L_k$ avec $(\lambda_1,...,\lambda_n) \in \mathbb K^n$ \\ 
Pour $j \in [[1,n]]$ : \\ 
$P(a_j)= \sum_{k=1}^n \lambda_k L_k(a_j) $ \\ 
$P(a_j)= \sum_{k=1}^n \lambda_k \delta(a_j) $ \\ 
$P(a_j)= \lambda_j $ \\ 
unicité des candidats $(\lambda_1,..., \lambda_n)$ \\
{\bf Synthèse :} Pour $Q=P- \sum_{k=1}^n P(\alpha_k) L_k$ \\ 
d'une part : $P,L_1,...,L_n \in \mathbb K_{n-1} [X]$ donc $Q \in \mathbb K_{n-1} [X]$ \\ 
d'autre part : pour $j \in [[1,n ]]$ : \\ 
$Q(a_j)=P(a_j)-\underbrace{\sum_{k=1}^n P(a_k) L_k(a_j)}_{=P(a_j)}$ \\ 
$Q(a_j)=0$ \\ 
Ainsi $deg(Q)<n$ et Q possède au moins n racines distinctes donc $Q=0$, $P= \sum_{k=1}^n P(\alpha_k) L_k$ \\
\section{Dérivée de PQ}
\textcolor{green}{Propriété :}
\section{Formule de Taylor}
\section{Caractérisation différentielle de la multiplicité}
\end{document}