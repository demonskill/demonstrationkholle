\documentclass{article}
\renewcommand*\familydefault{\sfdefault}
\usepackage[utf8]{inputenc}
\usepackage[T1]{fontenc}
\setlength{\textwidth}{481pt}
\setlength{\textheight}{650pt}
\setlength{\headsep}{10pt}
\usepackage{amsfonts}
\usepackage[T1]{fontenc}
\usepackage{palatino}
\usepackage{calrsfs}
\usepackage{geometry}
\geometry{ left=3cm, top=2cm, bottom=2cm, right=2cm}
\usepackage{xcolor}
\usepackage{amsmath}
\usepackage{tikz,tkz-tab}
\usepackage{cancel}
\usepackage{pgfplots}
\usepackage{pstricks-add}
\usepackage{pst-eucl}
\usepackage{amssymb}
\usepackage{icomma}
\usepackage{listings}
\begin{document}
\title{Démonstration kholle 23}
\date{}
\maketitle
	\renewcommand{\thesection}{\Roman{section}}
	\setlength{\parindent}{1.5cm}
	\section{ Produit matriciel : associativité, matrices identités.}
	\textcolor{green}{Propriété :}
	$\forall (A,B,C) \in M_{np}(\mathbb K) \times M_{pr}(\mathbb K) \times M_{rs}(\mathbb K), A(BC)=(AB)C$ \\
	\textcolor{red}{Démonstration :} \\
	$A=(a_{ij}){}_{\substack{1\leq i\leq n \\ 1\leq j\leq p}}$
	$B=(b_{ij}){}_{\substack{1\leq i\leq p \\ 1\leq j\leq r}}$
	 $C=(c_{ij}){}_{\substack{1\leq i\leq r \\ 1\leq j\leq s}}$
	 $BC=(d_{ij}){}_{\substack{1\leq i\leq p \\ 1\leq j\leq s}}$
	 $AB=(e_{ij}){}_{\substack{1\leq i\leq n \\ 1\leq j\leq r}}$ \\
	 Pour $1 \leq i \leq n$ et $1 \leq j \leq s$ d'une part, le coefficient i-j de $A(BC)$ est :\\
	  $\sum_{k=1}^p a_{ik}d_{kj}$ \\
	 or $d_{kj}= \sum_{l=1}^r b_{kl} c_{lj}$ donc c'est : $\sum_{k=1}^p \sum_{l=1}^r a_{ik}b_{kl}c_{lj}$ \\
	 d'autre part le coefficient de i-j de $(AB)C$ est : \\
	 $\sum_{l=1}^r e_{il}c_{ej}$ \\
	 Or $e_{il}= \sum_{k=1}^p a_{ik} b_{kl}$ donc c'est : $\sum_{l=1}^r \sum_{k=1}^p a_{ik} b_{kl} c_{lj}$ \\
	 même valeur par intervertion des sommes rectangulaires. \\
	 \textcolor{green}{Propriété :} \\
	 $\forall A \in M_{np}(\mathbb K), \underbrace{I_n}_{n \times n} \underbrace{A_{}}_{n \times p} = \underbrace{A_{}}_{n \times p} \underbrace{I_p}_{p \times p} = \underbrace{A_{}}_{n \times p} $ \\
	 \textcolor{red}{Démonstration :} \\
	 coefficient  i-j de $I_nA$ : \\
	 $\sum_{k=1}^n \delta_{ik}a_{kj}=a_{ij}$ \\
	 pour $A I_p :$ $\sum_{k=1}^p a_{ik} \delta_{kj}=a_{ij}$
	\section{ Matrice d'ordre 2 : caractérisation de l'inversibilité et formule pour l'inverse}
	\textcolor{green}{Propriété :} \\
	La matrice $A=\begin{pmatrix}
a & b \\
c & d
\end{pmatrix}$ est inversible si et seulement si $ad-bc \neq 0$ \\
 on a alors : $A^{-1}=\frac{1}{ad-bc}\begin{pmatrix}
d & -b \\
-c & a
\end{pmatrix}$ \\
\textcolor{red}{Démonstration :} \\
si $ad-bc \neq 0$ alors : \\
$\begin{pmatrix}
a & b \\
c & d
\end{pmatrix} \frac{1}{ad-bc}\begin{pmatrix}
d & -b \\
-c & a
\end{pmatrix}$  \\
$=\frac{1}{ad-bc}
\begin{pmatrix}
ad-bc & -ab+ab \\
cd-dc & ad-bc
\end{pmatrix}$ \\
$=I_2$ de même $\frac{1}{ad-bc}\begin{pmatrix} d & -b \\ -c & a \end{pmatrix}
\begin{pmatrix} a &b \\ c & d  \end{pmatrix} =I_2 $ \\
	si $ad-bc=0$ alors \\
	$\begin{pmatrix} a & b \\ c & d \end{pmatrix} \begin{pmatrix} a & -b \\ -c & a \end{pmatrix}=0_2$ \\
	si A était inversible on aurait : \\
	$A^{-1} A \times \begin{pmatrix} d & -b \\ -c & a \end{pmatrix} =A^{-1} \times 0_2 $ \\
	$\begin{pmatrix} d & -b \\ -c & a \end{pmatrix}=0_2$ \\
	$a=b=c=d=0$ donc $A=0_2$ contradiction avec A inversible
	\section{Une matrice carrée A est inversible si, et seulement si, pour toute colonne C, il existe une unique colonne X telle que $AX=C$}
	\textcolor{green}{Propriété :} \\
	Soit $A \in M_n(\mathbb K)$ on a les équivalences : \\
	\textcolor{green}{1)} $A \in GL_n(\mathbb K)$ \\
	\textcolor{green}{2)} Pour toute matrice colonne $C \in M_{n,1}(\mathbb K)$, il existe une unique solution $X \in M_{n,1}(\mathbb K)$ tel que $AX=C$ \\
	\textcolor{green}{3)} Tout système dont la matrice des coefficient est A possède une unique solution \\
	\textcolor{red}{Démonstration :} \\
	\textcolor{green}{$3 \Leftrightarrow 2$ } est immédiate. On transpose le système en écriture matricielle. \\
	\textcolor{green}{$1 \Rightarrow 3$} \\
	On cherche $X \in M_{n,1}(\mathbb K)$ tel que  $AX=C$ \\
	{\bf Analyse :} si une matrice $X_0$ convient alors comme A est inversible on aurait $X_0=A^{-1}C$ d'où l'unicité de $X_0$ \\
	{\bf Synthèse :} vérifions que $A^{-1}C$ convient : \\
	$A(A^{-1}C)= (A A^{-1})C=I_n C =C$ il convient \\
	\textcolor{green}{ $2 \rightarrow 1$} \\
	On va construire une matrice B tel que $AB=I_n$. \\
	On vérifiera $BA=I_n$ pour $j \in [[1,n]],$ notons $E_j \in M_{n,1}(\mathbb K)$ la j-ième colonne de $I_n$ \\
	soit $B_j \in M_{n,1}$ telle que $AB_j=E_j$ ($B_j$ existe et est unique par \textcolor{green}{2}) considérons $B \in M_n(\mathbb K)$ avec $B_j$ l'ensemble des colonnes de B : \\
	$B= \begin{pmatrix}
B_1 & B_2 & ... & B_n \\
	\end{pmatrix}$ \\
	alors $AB= \begin{pmatrix}
	AB_1 & AB_2 & ... & AB_n\end{pmatrix}=I_n$ (propriété au calcul matriciel par colonne) \\
	pour $j \in [[1,n]] ;$ on note note $A_j$ la j-ième colone de A \\
	par définition du produit matriciel : $AE_j=A_j$ \\
	Par ailleurs, on sais que $AB=I_n$ donc $(AB)A_j=A_j$ et $A(BA_j)=A_j$ \\
	donc par unicité $BA_j=E_j$ et par propriété du calcul matriciel : \\
	$BA=I_n$
	\section{Transposition : produit, inverse $M_n(\mathbb K)= S_n(\mathbb K) \oplus AS_n(\mathbb K)$}
	\textcolor{green}{Propriété :} \\
	Soit $A \in M_{np}( \mathbb K)$ et $B \in  M_{pr}(\mathbb K)$. Alors : \\
	 \indent $ \underbrace{{}^t(\underbrace{AB}_{n \times r})}_{r \times n}=\underbrace{\underbrace{{}^t B}_{r \times p}  \underbrace{{}^tA}_{p \times n}}_{r \times n}$ \\
	 \textcolor{red}{Démonstration :} \\
	 $A=(a_{ij}){}_{\substack{1\leq i\leq n \\ 1\leq j\leq p}}$,
	 $B=(b_{ij}){}_{\substack{1\leq i\leq p \\ 1\leq j\leq r}}$ \\
	 $AB=(c_{ij}){}_{\substack{1\leq i\leq n \\ 1\leq j\leq r}}$
	 ${}^tB{}^tA=(d_{ij}){}_{\substack{1\leq i\leq r \\ 1\leq j\leq n}}$ \\ \\
	 ${}^tA=(a'_{ij}){}_{\substack{1\leq i\leq p \\ 1\leq j\leq n}}$ avec $a'_{ij}=a_{ji}$
	 ${}^t B=(b'_{ij}){}_{\substack{1\leq i\leq r \\ 1\leq j\leq p}}$ avec $b'_{ij}=b_{ji}$
	 ${}^t(AB)=(c'_{ij}){}_{\substack{1\leq i\leq r \\ 1\leq j \leq n}}$ avec  $c'_{ij}=c_{ji}$ \\
	 Soit $(i,j) \in [[1,r]] \times [[1,n]]$ \\
	 $d_{ij} = \sum_{k=1}^p b'_{ik}a'_{kj}$ par définition de ${}^t B {}^t A$ \\
	 $d_{ij} =\sum_{k=1}^p b_{ki}a_{jk}$ déf de ${}^t A$ et ${}^tB$ \\
	 $d_{ij}=c_{ji}$ définition de $AB$ \\
	 $d_{ij}=c'_{ij}$ définition de ${}^t (AB)$ \\
\textcolor{green}{Propriété :} \\
Soit $A \in GL_n ( \mathbb K), {}^tA \in GL_n(\mathbb K)$ et $({}^tA)^{-1}={}^t(A^{-1})$ \\
\textcolor{red}{Démonstration :} \\
$AA^{-1}=A^{-1}A=I_n$ transposons : \\
${}^t (A^{-1}){}^tA= {}^t A {}^t A^{-1}=I_n$ car ${}^t I_n=I_n$ \\
donc ${}^t A \in Gl_n( \mathbb K)$ d'inverse ${}^t (A^{-1})$ \\
\textcolor{green}{Propriété :}  \\
$M_n(\mathbb K)= S_n(\mathbb K) \oplus AS_n(\mathbb K)$ \\
\textcolor{red}{Démonstration :} \\
Soit $s :M_n(\mathbb K) \rightarrow M_n (\mathbb K)$ \\
$A \mapsto {}^t A$ \\
Alors s est une involution linéaire ($s^2=Id_{M_n(\mathbb K)}$) donc : \\
$M_n (\mathbb K)= \underbrace{Ker(s-Id_{M_n(\mathbb K)})}_{S_n(\mathbb K)} \oplus \underbrace{Ker(s+ Id_{M_n(\mathbb K)})}_{AS_n(\mathbb K)}$
	\section{Trace : définition et propriété $tr(AB)=tr(BA)$.}
	\textcolor{red}{Définition :} \\
	soit $A=(a_{ij}){}_{\substack{1\leq i,j\leq n}}\in M_n(\mathbb K)$ \\
	On pose $tr(A)=\sum_{i=1}^n a_{ii} \in \mathbb K$ trace de A \\
	\textcolor{green}{Propriété :} \\
	$\forall A \in M_{np}(\mathbb K), \forall B \in M_{pn}(\mathbb K),tr(\underbrace{AB}_{n \times n})=tr(\underbrace{BA}_{p \times p})$ \\
	\textcolor{red}{Démonstration :} \\
	$A=(a_{ij}){}_{\substack{1\leq i\leq n \\ 1\leq j\leq p}}$,
	$B=(b_{ij}){}_{\substack{1\leq i\leq p \\ 1\leq j\leq n}}$ \\ \\
	$AB=(c_{ij}){}_{\substack{1\leq i,j\leq n}}$,
	$BA=(d_{ij}){}_{\substack{1\leq i,j\leq p}}$ \\
	$tr(AB)= \sum_{i=1}^n c_{ii}$ \\
	$tr(AB)= \sum_{i=1}^n \sum_{k=1}^p a_{ik}b_{ki}$ \\
	$tr(BA)= \sum_{i=1}^p d_{ii}$ \\
	$tr(BA)= \sum_{i=1}^p \sum_{k=1}^n b_{ik}a_{ki}$ \\
	$tr(BA)= \sum_{k=1}^n \sum_{i=1}^p a_{ki}b_{ik}$ \\
	On échange les rôles des indices i et k : \\
	$tr(BA)= \sum_{i=1}^n \sum_{k=1}^p a_{ik}b_{ki}=tr(AB)$
	\section{L'application $\Phi : A \in M_{n,p} \longmapsto (\Phi_A : M \mapsto tr({}^tAM))\in \mathcal L(M_{n,p}(\mathbb K), \mathbb K)$ est un isomorphisme}
\textcolor{green}{Propriété :} \\
on fixe $(n,p) \in \mathbb (N^*)^2$ \\
\textcolor{green}{1)} Pour $A \in M_{np}(\mathbb K)$, l'application : \\
$\phi_A :M_{np}(\mathbb K) \rightarrow \mathbb K$ \\
$M \mapsto tr(\underbrace{{}^t AM}_{p \times p})$ \\
est unes forme linéaire sur $M_{np}(\mathbb K)$ \\
\textcolor{green}{2)} L'application : \\
$\Phi : M_{np} \rightarrow \mathcal L(M_{np}(\mathbb K), \mathbb K)$ \\
$A \mapsto \phi_A$ est un isomorphisme \\
\textcolor{red}{Démonstration :} \\
\textcolor{green}{1)} $A \in M_{np}( \mathbb K)$ fixé \\
Pour $M_1$ et $M_2 \in M_{np} (\mathbb K)$ et $\lambda \in K$ \\
$\phi_{A}(M_1+ \lambda M_2) = tr({}^tA (M_1 + \lambda M_2))$ \\
$\phi_{A}(M_1+ \lambda M_2) = tr({}^tAM_1 + \lambda {}^t AM_2)$ \\
$\phi_{A}(M_1+ \lambda M_2) = tr({}^tAM_1) + \lambda tr({}^t AM_2)$  \\
$\phi_{A}(M_1+ \lambda M_2) =\phi_A(M_1)+ \lambda \phi_A (M_2)$ \\
\textcolor{green}{2)}Montrons que { \boldmath $ \Phi \in  \mathcal L(M_{np}(\mathbb K),\mathcal L(M_{np}(\mathbb K), \mathbb K))$} \\
Soit $(A,B) \in M_{np}(\mathbb K)$ et $\lambda \in \mathbb K$ \\
Pour $M \in M_{np}(\mathbb K)$ : \\
$ \phi_{A+\lambda B}= tr ({}^t(A+ \lambda B)M)$ \\
$ \phi_{A+\lambda B}= tr ({}^tAM+ \lambda {}^tBM)$ \\
$\phi_{A+\lambda B}= tr ({}^tAM)+ \lambda tr({}^tBM) $ \\
$\phi_{A+\lambda B}= \phi_A (M) + \lambda \phi_B (M)$ \\
ainsi : $\forall M \in M_{np}(\mathbb K), \phi_{A + \lambda B}(M)= \phi_A (M) + \lambda \phi_B (M)$ \\
c'est-à-dire : $\phi_{A+ \lambda B}=\phi_A + \lambda \phi_B$ \\
Montrons que {\bf \boldmath $\Phi$ injective}, c'est-à-dire $Ker(\Phi)= \lbrace 0_{np}\rbrace $ \\
$Ker( \Phi) \supset  \lbrace 0_{np}\rbrace$ toujours vraie \\
$Ker( \Phi) \subset  \lbrace 0_{np}\rbrace$ Soit $A \in M_{np}(\mathbb K)$ tel que : $\Phi(A)=\tilde{0}$ \\
$\forall M \in M_{np}(\mathbb K), \phi_A(M)=0$ \\
en particulier : \\
pour $1 \leq i \leq n$ et $1 \leq j \leq p$ \\
$\phi_A(E_{ij})=0$ \\
donc $tr({}^t A E_{ij})=0$ \\
${}^t A E_ij = \begin{pmatrix}
[[1, j-1]] & j & [j+1, p] \\
0	& a_{i,1} &0 \\
0	& . & 0\\
0	& . & 0\\
0	& . & 0\\
0	& a_{i,p} & 0 \\
\end{pmatrix} \in M_p( \mathbb K) $ \\
$tr({}^t A E_{ij})= a_{ij}$ \\
on a donc : \\
$\forall (i,j) \in [[1,n]] \times [[1,p]], a_{ij}= 0$ \\
$A= 0_{np}$ \\
{ \bf Théorème du rang :} \\
$dim(M_{np})=\underbrace{dim(Ker(\Phi))}_{=0}+ dim (Im(\Phi))$ \\
$dim(M_{np})= rg(\Phi)$ \\
or : $Im( \phi) \subset  \mathcal L(M_{np}(\mathbb K), \mathbb K)$ \\
et $dim( \mathcal L(M_{np}(\mathbb K), \mathbb K))= dim( M_{np}(\mathbb K)) \times 1 =np=rg(\Phi)$ \\
donc $Im(\phi)=  \mathcal L(M_{np}(\mathbb K), \mathbb K)$
\end{document}
