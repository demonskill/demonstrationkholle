\documentclass{article}
\renewcommand*\familydefault{\sfdefault}
\usepackage[utf8]{inputenc}
\usepackage[T1]{fontenc}
\setlength{\textwidth}{481pt}
\setlength{\textheight}{650pt}
\setlength{\headsep}{10pt}
\usepackage{amsfonts}
\usepackage[T1]{fontenc}
\usepackage{palatino}
\usepackage{calrsfs}
\usepackage{geometry}
\geometry{ left=3cm, top=2cm, bottom=2cm, right=2cm}
\usepackage{xcolor}
\usepackage{amsmath}
\usepackage{tikz,tkz-tab}
\usepackage{cancel}
\usepackage{pgfplots}
\usepackage{pstricks-add}
\usepackage{pst-eucl}
\usepackage{amssymb}
\usepackage{icomma}
\usepackage{listings}
\begin{document}
\title{Démonstration kholle 23}
\date{}
\maketitle
	\renewcommand{\thesection}{\Roman{section}}
	\setlength{\parindent}{1.5cm}
	\section{ Produit matriciel : associativité, matrices identités.}
	\textcolor{green}{Propriété :}
	$\forall (A,B,C) \in M_{np}(\mathbb K) \times M_{pr}(\mathbb K) \times M_{rs}(\mathbb K), A(BC)=(AB)C$ \\
	\textcolor{red}{Démonstration :} \\
	$A=(a_{ij}){}_{\substack{1\leq i\leq n \\ 1\leq j\leq p}}$
	$B=(b_{ij}){}_{\substack{1\leq i\leq p \\ 1\leq j\leq r}}$
	 $C=(c_{ij}){}_{\substack{1\leq i\leq r \\ 1\leq j\leq s}}$
	 $BC=(d_{ij}){}_{\substack{1\leq i\leq p \\ 1\leq j\leq s}}$
	 $AB=(e_{ij}){}_{\substack{1\leq i\leq n \\ 1\leq j\leq r}}$
	 Pour $1 \leq i \leq n$ et $1 \leq j \leq s$ d'une part, le coefficient i-j de $A(BC)$ est :\\
	  $\sum_{k=1}^n a_{ik}d_{kj}$ \\
	 or $d_{kj}= \sum_{l=1}^r b_{kl} c_{lj}$ donc c'est : $\sum_{k=1}^p \sum_{l=1}^r a_{ik}b_{kl}c_{lj}$ \\
	 d'autre part le coefficient de i-j de $(AB)C$ est : \\
	 $\sum_{l=1}^r e_{il}c_{ej}$ \\
	 Or $e_{il}= \sum_{k=1}^p a_{ik} b_{kl}$ donc c'est : $\sum_{l=1}^r \sum_{k=1}^p a_{ik} b_{kl} c_{lj}$ \\
	 même valeur par intervertion des sommes rectangulaires. \\
	 \textcolor{green}{Propriété :}
	 $\forall A \in M_{np}(\mathbb K), \underbrace{I_n}_{n \times n} \underbrace{A_{}}_{n \times p} = \underbrace{A_{}}_{n \times p} \underbrace{I_p}_{p \times p} = \underbrace{A_{}}_{n \times p} $ \\
	 \textcolor{red}{Démonstration :} \\
	 coefficient  i-j de $I_nA$ : \\
	 $\sum_{k=1}^n \delta_{ik}a_{kj}=a_{ij}$ \\
	 pour $A I_p :$ $\sum_{k=1}^p a_{ik} \delta_{kj}=a_ij$
	\section{ Matrice d'ordre 2 : caractérisation de l'inversibilité et formule pour l'inverse}
	\textcolor{green}{Propriété :} \\
	La matrice $A=\begin{pmatrix}
a & b \\
c & d
\end{pmatrix}$ est inversible si et seulement si $ad-bc \neq 0$ et on a alors : $A^{-1}=\frac{1}{ad-bc}\begin{pmatrix}
d & -b \\
-c & a
\end{pmatrix}$ \\
\textcolor{red}{Démonstration :} \\
si $ad-bc \neq 0$ alors : \\
$\begin{pmatrix}
a & b \\
c & d
\end{pmatrix} \frac{1}{ad-bc}\begin{pmatrix}
d & -b \\
-c & a
\end{pmatrix}$  \\
$=\frac{1}{ad-bc}
\begin{pmatrix}
ad-bc & -ab+ab \\
cd-dc & ad-bc
\end{pmatrix} \begin{pmatrix}
a & b \\
c & d
\end{pmatrix}$ \\
$=I_2$ de même $\frac{1}{ad-bc}\begin{pmatrix} d & -b \\ -c & a \end{pmatrix}
\begin{pmatrix} a &b \\ c & d  \end{pmatrix} =I_2 $ \\
	si $ad-bc=0$ alors \\
	$\begin{pmatrix} a & b \\ c & d \end{pmatrix} \begin{pmatrix} a & -b \\ -c & a \end{pmatrix}=0_2$ \\
	si A était inversible on aurait : \\
	$Aç{-1} A \times \begin{pmatrix} d & -b \\ -c & a \end{pmatrix} =A^{-1} \times 0_2 $ \\
	$\begin{pmatrix} d & -b \\ -c & a \end{pmatrix}=0_2$ \\
	$a=b=c=d=0$ donc $A=0_2$ contradiction avec A inversible
\end{pmatrix}
	\section{Une matrice carrée A est inversible si, et seulement si, pour toute colonne C, il existe une unique colonne X telle que $AX=C$}
	\textcolor{green}{Propriété :} \\
	Soit $A \in M_n(\mathbb K)$ on a les équivalences : \\
	\textcolor{green}{1)} $A \in GL_n(\mathbb K)$ \\
	\textcolor{green}{2)} Pour toute matrice colonne $C \in M_{n,1}(\mathbb K)$, il existe une unique solution $X \in M_{n,1}(\mathbb K)$ tel que $AX=C$ \\
	\textcolor{green}{3)} Tout système dont la matrice des coefficient est A possède une unique solution \\
	\textcolor{red}{Démonstration :} \\
	\textcolor{green}{$3 \Leftrightarrow 2$ } est immédiate. On transpose le système en écriture matricielle. \\
	\textcolor{green}{$1 \Rightarrow 3$} \\
	On cherche $X \in M_{n,1}(\mathbb K)$ tel que  $AX=C$ \\
	{\bf Analyse :} si une matrice $X_0$ convient alors comme A est inversible on aurait $X_0=A^{-1}C$ d'où l'unicité de $X_0$ \\
	{\bf Synthèse :} vérifions que $A^{-1}C$ convient : \\
	$A(A^{-1}C)= (A A^{-1})C=I_n C =C$ il convient \\
	\textcolor{green}{ $2 \rightarrow 1$} \\
	On va construire une matrice B tel que $AB=I_n$. On vérifiera $BA=I_n$ pour $j \in [[1,n]],$ notons $E_j \in M_{n,1}(\mathbb K)$ la j-ième colonne de $I_n$ soit $B_j \in M_{n,1}$ telle que
	$AB_j=E_j$ ($B_j$ existe et est unique par \textcolor{green}{2}) considérons $B \in M_n(\mathbb K)$ avec $B_j$ l'ensemble des colonnes de B : \\
	$B= \begin{pmatrix}
B_1 & B_2 & ... & B_n \\
	\end{pmatrix}$ \\
	alors $AB= \begin{pmatrix}
	AB_1 & AB_2 & ... & AB_n\end{pmatrix}=I_n$ (propriété au calcul matriciel par colonne) \\
	pour $j \in [[1,n]] ;$ on note note $A_j$ la j-ième colone de A \\
	par définition du produit matriciel : $AE_j=A_j$ \\
	Par ailleurs, on sais que $AB=I_n$ donc $(AB)A_j=A_j$ et $A(BA_j)=A_j$
	donc par unicité $BA_j=E_j$ et par propriété du calcul matriciel : \\ $BA=I_n$
	\section{Transposition : produit, inverse $M_n(\mathbb K)= S_n(\mathbb K) \oplus AS_n(\mathbb K)$}
	\section{Trace : définition et propriété $tr(AB)=tr(BA)$.}
	\section{L'application $\Phi : A \in M_{n,p} \longmapsto (\Phi_A : M \mapsto tr({}^tAM))\in \mathcal L(M_{n,p}(\mathbb K), \mathbb K)$ est un isomorphisme}
\end{document}
