\documentclass{article}
\renewcommand*\familydefault{\sfdefault}
\usepackage[utf8]{inputenc}
\usepackage[T1]{fontenc}
\setlength{\textwidth}{481pt}
\setlength{\textheight}{650pt}
\setlength{\headsep}{10pt}
\usepackage{amsfonts}
\usepackage[T1]{fontenc}
\usepackage{palatino}
\usepackage{calrsfs}
\usepackage{geometry}
\geometry{ left=3cm, top=2cm, bottom=2cm, right=2cm}
\usepackage{xcolor}
\usepackage{amsmath}
\usepackage{tikz,tkz-tab}
\usepackage{cancel}
\usepackage{pgfplots}
\usepackage{pstricks-add}
\usepackage{pst-eucl}
\usepackage{amssymb}
\usepackage{icomma}
\usepackage{listings}
\begin{document}
\title{Démonstration kholle 23}
\date{}
\maketitle
	\renewcommand{\thesection}{\Roman{section}}
	\setlength{\parindent}{1.5cm}
	\section{ Produit matriciel : associativité, matrices identités.}
	\section{ Matrice d'ordre 2 : caractérisation de l'inversibilité et formule pour l'inverse}
	\section{Une matrice carrée A est inversible si, et seulement si, pour toute colonne C, il existe une unique colonne X telle que $AX=C$}
	\section{Transposition : produit, inverse $M_n(\mathbb K)= S_n(\mathbb K) \oplus AS_n(\mathbb K)$}
	\section{Trace : définition et propriété $tr(AB)=tr(BA)$.}
	\section{L'application $\Phi : A \in M_{n,p} \longmapsto (\Phi_A : M \mapsto tr({}^tAM))\in \mathcal L(M_{n,p}(\mathbb K), \mathbb K)$ est un isomorphisme}
\end{document}
