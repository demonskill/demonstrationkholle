\documentclass{article}
\renewcommand*\familydefault{\sfdefault}
\usepackage[utf8]{inputenc}
\usepackage[T1]{fontenc}
\setlength{\textwidth}{481pt}
\setlength{\textheight}{650pt}
\setlength{\headsep}{10pt}
\usepackage{amsfonts}
\usepackage[T1]{fontenc}
\usepackage{palatino}
\usepackage{calrsfs}
\usepackage{geometry}
\geometry{ left=3cm, top=2cm, bottom=2cm, right=2cm}
\usepackage{xcolor}
\usepackage{amsmath}
\usepackage{tikz,tkz-tab}
\usepackage{cancel}
\usepackage{pgfplots}
\usepackage{pstricks-add}
\usepackage{pst-eucl}
\usepackage{amssymb}
\usepackage{icomma}
\usepackage{listings}
\begin{document}
\title{Démonstration kholle 22}
\date{}
\maketitle
	\renewcommand{\thesection}{\Roman{section}}
	\setlength{\parindent}{1.5cm}
\section{Théorème de la base incomplète (cas $\mathcal L \subset \mathcal G$).}
\textcolor{green}{Théorème de la base incomplète :} \\
Soit  E un K-espace vectoriel de dimension finie : \\
Soit $ \mathcal G$ une famille génératrice finie de E et $\mathcal L \subset \mathcal G$ tel que $ \mathcal L$ soit libre. \\
Alors il existe une base $\mathcal B$ de E tel que $\mathcal L \subset \mathcal B \subset \mathcal G$ \\
\textcolor{red}{Démonstration :} \\
Posons : $\Omega= \lbrace \mathcal H$ famille libre de E : $ \mathcal L \subset \mathcal H \subset \mathcal G \rbrace$ \\
$A= \lbrace Card(\mathcal H) : \mathcal H \in \Omega \rbrace$ \\
A est une partie de $\mathbb N$ majorée par $Card(\mathcal G)$ et $A \neq \emptyset$ car $ \mathcal L \in \Omega $ donc $card(\mathcal L) \in A$ \\
Soit $p=max(A)$ et $B \in \Omega$ tel que $Card(B)=p$. \\
On a bien $\mathcal L \subset \mathcal B \subset \mathcal G$ et $\mathcal B$ libre car $\mathcal B \in \Omega$ \\
Supposons que B n'engendre pas E : \\
Si tous les vecteurs de $\mathcal G$ appartiennent à $Vect(\mathcal B)$ : \\
$E=Vect(\mathcal G) \subset Vect(\mathcal B)$, ce qu'on a exclu. \\
Soit $\vec x \in \mathcal G$ tel que $\vec x \notin Vect( \mathcal B )$ et $\mathcal M= \mathcal B \cup \lbrace \vec x \rbrace$ alors : \\
$\mathcal L \subset \mathcal M \subset \mathcal G$ et $ \mathcal M$ libre car $ \mathcal B$ libre et $x \notin Vect(\mathcal B)$ \\
donc $\mathcal M \in \Omega$ et $Card(M)>max(A)$ absurde
\section{Lemme de Steinitz.}
\textcolor{green}{Lemme de Steinitz :} \\
Dans un K-espace vectoriel engendré par n vecteurs, toute famille d'au moins n+1 vecteurs est liée \\
\textcolor{red}{Démonstration :} \\
récurrence sur n : \\
{\bf Initialisation :}  n=0, l'espace est noté $ \lbrace  \vec 0 \rbrace $ \\
Seule la famille ($\vec 0$) convient qui est liée car $\underbrace{1}_{\neq 0}\cdot \vec 0= \vec 0$ \\
{\bf Hérédité : } soit $n \in \mathbb N$ tel que les propriétés soit vraie au rang n : \\
Soit E engendré par n+1 vecteurs \\
$E=Vect(\vec g_1,...,\vec g_{n+1})$ \\
Soit une famille de n+2 vecteurs : $(\vec u_1,...,\vec u_{n+2})$ \\
Il existe des scalairs $\alpha_{i,j}$ ($1 \leq i \leq n+2$, $1 \leq j \leq n+1$) tel que : \\
$\forall i [[1,n+2]], \vec u_i= \sum_{j=1}^{n+1} \alpha_{i,j} \vec g_j$ car $(\vec g_1,...,\vec g_{n+1})$ engendre E : \\
Notons : \\
$\vec v_i=\sum_{k=1}^n \alpha_{i,j} \vec g_j (1 \leq i \leq n)$ \\
$\beta_i=\alpha_{i,n+1}$ \\
$F=Vect(\vec g_1,...,\vec g_n)$ \\
Alors : $\forall i \in [[1,n+2]], \vec u_i=\vec v_i + \beta_i \vec g_{n+1}$ \\
{\boldmath \bf $1^{er}$ cas :} Si $\beta_1=...=\beta_{n+2}=0$ \\
Alors : $\forall i \in [[1,n+2]], \vec u_i=\vec v_i \in F$ ainsi $(\vec u_1,..., \vec u_{n+2})$ est une famille d'au moins n+1 vecteurs de F qui est engendré par n vecteurs per hypothèse de récurrence, elle est liée. \\
{\boldmath $2^{er}$ \bf cas :}$\exists i_0 \in [[1,n+2]], \beta_{i_0}\neq 0$sans perte de généralité supposons $\beta_{n+2}\neq 0$ \\
On a donc: $\vec g_{n+1}= \frac{1}{\beta_{n+2}}(\vec u_{n+2}-\vec v_{n+2})$ \\
Donc pour $1 \leq i \leq n+1$ : \\
$\vec u_i=\vec v_i+ \frac{\beta_i}{\beta_{n+2}}(\vec u_{n+2}-\vec v_{n+2})$ \\
Posons $\vec w_i= \vec u_i - \frac{\beta_i}{\beta_{n+2}}\vec u_{n+2} (1 \leq i \leq n+1)$ \\
Ainsi : $\forall i \in [[1,n+1]], \vec w_i=\vec v_i-\frac{\beta_i}{\beta_{n+2}} \vec v_{n+2} \in F $ \\
Par hypothèse de récurrence : \\
$(\vec w_1,...,\vec w_{n+1})$ est liée \\
ainsi $\exists (\mu_1,...,\mu_{n+1}) \in \mathbb K^{n+1}, \sum_{i=1}^{n+1} \mu_i \vec w_i= \vec 0$ et $(\mu_1,...,\mu_{n+1}) \neq (0,...,0)$ \\
c'est-à-dire : $\mu_1 \vec u_1 +... +\mu_{n+1} \vec u_{n+1}+(- \sum_{i=1}^{n+1} \frac{\mu_i \beta_i}{\beta_{n+2}}) \vec u_{n+2}= \vec 0$ \\
Comme au moins l'un des n+1 premiers coefficients $\neq 0$, \\
cela montre que : $(\vec u_1,..., \vec u_{n+2})$ est liée.
\section{Conséquences du lemme de Steinitz : les bases sont finies et de même cardinal, cardinaux des familles libres et cas d'égalité.}
\textcolor{green}{Propriétés :} \\
Soit E un K-espace vectoriel de dimension finie. Toutes ses bases sont finies et ont toutes même cardinal. Ce cardinal est la dimension de E, notéé dim(E) \\
\textcolor{red}{Démonstration :} \\
On sait qu'il existe une base finie $\mathcal B_0$. Soit $\mathcal B$ une base de E, elle est libre donc finie et $ \mathcal B_0$engendre E donc $Card(B) \leq card(B_0)$. De même : $B_0$ est libre et B génératrice de E donc $Card(B_0) \leq Card(B)$ on a donc $Card(\mathcal B)=Card(\mathcal B_0)$\\
\textcolor{green}{Propriété :} \\
Soit $ \mathcal L$ une famille libre de E alors : \\
\textcolor{green}{1)} $\mathcal L$ est finie et $Card(\mathcal L) \leq dim(E)$ \\
\textcolor{green}{2)} $Card(\mathcal L)=dim(E) \Longleftrightarrow \mathcal L$ base de E \\
\textcolor{red}{Démonstration :} \\
\textcolor{green}{1)} Soit $\mathcal B$ une base de E. Lemme de Steinitz : $\mathcal L$ libre et $\mathcal B$ génératrice finie \\
donc $\mathcal L$ finie et $Card(\mathcal L) \leq \underbrace{Card(\mathcal B)}_{=dim(E)}$ \\
\textcolor{green}{2)} $\Rightarrow$ definition de dim(E) par le fait que toutes les bases ont même cardinal \\
$\Leftarrow$ Notons $n=dim(E)$ \\
Si n=0 : $E=\lbrace 0 \rbrace$ et $\mathcal L=()$ \\
Si $n \geq 1$ : Notons $\mathcal L=(\vec x_1,..., \vec x_n)$ \\
Si $\mathcal L$ n'engendrait pas E: \\
Considérons $\vec x_{n+1} \in E$ tel que $\vec x_{n+1} \notin Vect(\vec x_1,...,\vec x_n)$ \\
Alors $(\vec x_1,...,\vec x_{n+1})$ est encore libre mais son cardinal est $n+1> dim(E)$: contradiction \\
Donc $\mathcal L$ engendre E et est libre c'est donc une base E
\section{Sous-espaces : inégalité des dimensions et cas d'égalité. }
\textcolor{green}{Propriété :} \\
Soit E K-espace vectoriel, F sous-espace vectoriel de E \\
\textcolor{green}{1)} F est de dimension finie et $dim(F) \leq dim(E)$ \\
\textcolor{green}{2)} $dim(F)=dim(E) \Longleftrightarrow F=E$ \\
\textcolor{red}{Démonstration :} \\
\textcolor{green}{1)} $\Omega=\lbrace \mathcal L :$ famille libre de F $\rbrace$ \\
$A=\lbrace Card(\mathcal L) : \mathcal L \in \Omega \rbrace$. Pour $\mathcal L \in \Omega  : \mathcal L$ est aussi une famille libre de E donc $\mathcal L$ est finie et $Card(\mathcal L) \leq dim(E)$ \\
Ainsi : $A \subset \mathbb N$ et majoré par $dim(E)$ $A \neq \subset$  car $0 \in A$(car $()\in \Omega$) donc A possède un maximum p : \\
Soit $\mathcal L_0 \in \Omega$ tel que $Card(\mathcal L_0) = p $ : \\
On a : $\mathcal L_0$ libre car $\mathcal L \in \Omega$, $Vect(\mathcal L_0) \subset F$ par définition de $ \Omega $ \\
Si $Vect(\mathcal L_0) \subsetneq F$ : \\
Soit $\vec x \in  F \backslash Vect (\mathcal L_0)$ : \\
Alors $\mathcal L_0 \cup (\vec x)$ est une famille libre de F donc $\mathcal L \cup (\vec x) \in \Omega$ mais $Card(\mathcal L_0 \cup (\vec x))=p+1>p$ contradiction \\
Ainsi : $Vect(\mathcal L_0)=F$ \\
Conclusion : $\mathcal L_0$ est une base de F donc: F est de dimension finie et $dim(F)=p \leq dim(E)$ car $dim(E)$ est un majorant de A. \\
\textcolor{green}{2)} Soit $ \mathcal B$ une base de F \\
En particulier : $ \mathcal B$ est une famille libre de F donc $ \mathcal B$ est une famille de E. \\
De plus: $Card(\mathcal B)=dim(F)$ car $\mathcal B$ base de F ^^
$Card(\mathcal B)=dim(E)$ par hypothèse donc $\mathcal B$ base de E \\
$F=Vect(\mathcal B)=E$ \\
\section{Si $E= F \oplus G, dim(E)=dim(F)+dim(G)$}
\section{Existence de supplémentaires, formule de Grassmann.}
\section{Dimension de $E \times F$, de $\mathcal L(E,F)$.}
\section{Détermination d'une application linéaire par l'image d'une base (énoncé complet + démonstration de la caractérisation de l'injectivité).}
\section{Théorème du rang (avec énoncé et démonstration du lemme )}
\end{document}
