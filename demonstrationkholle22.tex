\documentclass{article}
\renewcommand*\familydefault{\sfdefault}
\usepackage[utf8]{inputenc}
\usepackage[T1]{fontenc}
\setlength{\textwidth}{481pt}
\setlength{\textheight}{650pt}
\setlength{\headsep}{10pt}
\usepackage{amsfonts}
\usepackage[T1]{fontenc}
\usepackage{palatino}
\usepackage{calrsfs}
\usepackage{geometry}
\geometry{ left=3cm, top=2cm, bottom=2cm, right=2cm}
\usepackage{xcolor}
\usepackage{amsmath}
\usepackage{tikz,tkz-tab}
\usepackage{cancel}
\usepackage{pgfplots}
\usepackage{pstricks-add}
\usepackage{pst-eucl}
\usepackage{amssymb}
\usepackage{icomma}
\usepackage{listings}
\begin{document}
\title{Démonstration kholle 22}
\date{}
\maketitle
	\renewcommand{\thesection}{\Roman{section}}
	\setlength{\parindent}{1.5cm}
\section{Théorème de la base incomplète (cas $\mathcal L \subset \mathcal G$).}
\section{Lemme de Steinitz.}
\section{Conséquences du lemme de Steinitz : les bases sont finies et de même cardinal, cardinaux des familles libres et cas d'égalité.}
\section{Sous-espaces : inégalité des dimensions en cas d'égalité. }
\section{Si $E= F \oplus G, dim(E)=dim(F)+dim(G)$}
\section{Existence de supplémentaires, formule de Grassmann.}
\section{Dimension de $E \times F$, de $\mathcal L(E,F)$.}
\section{Détermination d'une application linéaire par l'image d'une base (énoncé complet + démonstration de la caractérisation de l'injectivité).}
\section{Théorème du rang (avec énoncé et démonstration du lemme )}
\end{document}
