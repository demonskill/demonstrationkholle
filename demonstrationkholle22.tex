\documentclass{article}
\renewcommand*\familydefault{\sfdefault}
\usepackage[utf8]{inputenc}
\usepackage[T1]{fontenc}
\setlength{\textwidth}{481pt}
\setlength{\textheight}{650pt}
\setlength{\headsep}{10pt}
\usepackage{amsfonts}
\usepackage[T1]{fontenc}
\usepackage{palatino}
\usepackage{calrsfs}
\usepackage{geometry}
\geometry{ left=3cm, top=2cm, bottom=2cm, right=2cm}
\usepackage{xcolor}
\usepackage{amsmath}
\usepackage{tikz,tkz-tab}
\usepackage{cancel}
\usepackage{pgfplots}
\usepackage{pstricks-add}
\usepackage{pst-eucl}
\usepackage{amssymb}
\usepackage{icomma}
\usepackage{listings}
\begin{document}
\title{Démonstration kholle 22}
\date{}
\maketitle
	\renewcommand{\thesection}{\Roman{section}}
	\setlength{\parindent}{1.5cm}
\section{Théorème de la base incomplète (cas $\mathcal L \subset \mathcal G$).}
\textcolor{green}{Théorème de la base incomplète :} \\
Soit  E un K-espace vectoriel de dimension finie : \\
Soit $ \mathcal G$ une famille génératrice finie de E et $\mathcal L \subset \mathcal G$ tel que $ \mathcal L$ soit libre. \\
Alors il existe une base $\matchcal B$ de E tel que $\mathcal L \subset \mathcal B \subset \mathcal G$ \\
\textcolor{red}{Démonstration :} \\
Posons : $\Omega= \lbrace \mathcal H$ famille libre de E : $ \mathcal L \subset \mathcal H \subset \mathcal G \rbrace$ \\
$A= \lbrace Card(\mathcal H) : \mathcal H \in \Omega \rbrace$ \\
A est une partie de $\mathbb N$ majorée par $Card(\mathcal G)$ et $A \neq \emptyset$ car $ \mathcal L \in \Omega $ donc $card(\mathcal L) \in A$ \\
Soit $p=max(A)$ et $B \in \Omega$ tel que $Card(B)=p$. \\
On a bien $\mathcal L \subset \mathcal B \subset \mathcal G$ et $\mathcal B$ libre car $\mathcal B \in \Omega$ \\
Supposons que B n'engendre pas E : \\
Si tous les vecteurs de $\mathcal G$ appartiennent à $Vect(\mathcal B)$ : \\
$E=Vect(\mathcal G) \subset Vect(\mathcal B)$, ce qu'on a exclu. \\
Soit $\vec x \in \mathcal G$ tel que $\vec x \notin Vect( \mathcal B )$ et $\mathcal M= \mathcal B \cup \lbrace \vec x \rbrace$ alors : \\
$\mathcal L \subset \mathcal M \subset \mathcal G$ et $ \mathcal M$ libre car $ \matchcal B$ libre et $x \notin Vect(\mathcal B)$ \\
donc $\mathcal M \in \Omega$ et $Card(M)>max(A)$ absurde
\section{Lemme de Steinitz.}
\textcolor{green}{Lemme de Steinitz :} \\
Dans un K-espace vectoriel engendré par n vecteurs, toute famille d'au moins n+1 vecteurs est liée \\
\textcolor{red}{Démonstration}
\section{Conséquences du lemme de Steinitz : les bases sont finies et de même cardinal, cardinaux des familles libres et cas d'égalité.}
\section{Sous-espaces : inégalité des dimensions en cas d'égalité. }
\section{Si $E= F \oplus G, dim(E)=dim(F)+dim(G)$}
\section{Existence de supplémentaires, formule de Grassmann.}
\section{Dimension de $E \times F$, de $\mathcal L(E,F)$.}
\section{Détermination d'une application linéaire par l'image d'une base (énoncé complet + démonstration de la caractérisation de l'injectivité).}
\section{Théorème du rang (avec énoncé et démonstration du lemme )}
\end{document}
