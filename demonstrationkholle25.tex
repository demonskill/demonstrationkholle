\documentclass{article}
\renewcommand*\familydefault{\sfdefault}
\usepackage[utf8]{inputenc}
\usepackage[T1]{fontenc}
\setlength{\textwidth}{481pt}
\setlength{\textheight}{650pt}
\setlength{\headsep}{10pt}
\usepackage{amsfonts}
\usepackage[T1]{fontenc}
\usepackage{palatino}
\usepackage{calrsfs}
\usepackage{geometry}
\geometry{ left=3cm, top=2cm, bottom=2cm, right=2cm}
\usepackage{xcolor}
\usepackage{amsmath}
\usepackage{tikz,tkz-tab}
\usepackage{cancel}
\usepackage{pgfplots}
\usepackage{pstricks-add}
\usepackage{pst-eucl}
\usepackage{amssymb}
\usepackage{icomma}
\usepackage{listings}
\begin{document}
\title{Démonstration kholle 25}
\date{}
\maketitle
\renewcommand{\thesection}{\Roman{section}}
	\setlength{\parindent}{1.5cm}
	\section{Toute forme n-linéaire alternée est antisymétrique.}
	\textcolor{green}{Propriété :} \\
	Soit $\phi \in A_n(E)$ \\
	$\forall (\vec x_1,...,\vec x_n) \in E^n, \forall \sigma \in S_n, \phi(\vec x_{\sigma(1)},...,\vec x_{\sigma(n)})=\epsilon(\sigma) \phi(\vec x_1,...,\vec x_n)$ \\
	On dit que $\phi$ est antisymétrique \\
	\textcolor{red}{Démonstration :} \\
	cas où $\sigma =(i \quad  j), i<j$ Comme $\phi$ est alternée : \\
	$\phi(\vec x_1,..., \vec x_i + \vec x_j,..., \vec x_i + \vec x_j,...,\vec x_n)=0$ \\
	$0=\phi(\vec x_1,..., \vec x_i,..., \vec x_i + \vec x_j,...,\vec x_n)+\phi(\vec x_1,...,  \vec x_j,..., \vec x_i + \vec x_j,...,\vec x_n) $ \\
	$0=\underbrace{\phi(\vec x_1,..., \vec x_i,..., \vec x_i ,...,\vec x_n)}_{=0}+\phi(\vec x_1,...,  \vec x_j,..., \vec x_i ,...,\vec x_n)+\phi(\vec x_1,..., \vec x_i,...,  \vec x_j,...,\vec x_n)+\underbrace{\phi(\vec x_1,...,  \vec x_j,..., \vec x_j,...,\vec x_n)}_{=0}$ \\
	$\phi(\vec x_1,..., \vec x_j,...,  \vec x_i,...,\vec x_n)=-\phi(\vec x_1,...,  \vec x_i,..., \vec x_j ,...,\vec x_n)$ \\
	Or $\epsilon(\sigma)=-1$ \\
	Pour $\sigma $ quelconque : \\
	$\sigma= \sigma_1 .... \sigma_m$ avec $\sigma_k$ transposition. \\
	$\phi(\vec x_{\sigma(1)},..., \vec x_{\sigma(n)})=\underbrace{\epsilon(\sigma_1)... \epsilon(\sigma_m)}_{\epsilon(\sigma)} \phi(\vec x_1,..., \vec x_m)$ \\
	$\phi(\vec x_{\sigma(1)},...,\vec x_{\sigma(n)})=\epsilon(\sigma) \phi(\vec x_1,...,\vec x_n)$
	\section{$det_{\mathcal B}$ est alternée et vaut 1 sur $\mathcal B$.}
	\textcolor{green}{Propriété :} \\
	Soit $\mathcal B$ une base de E . \\
	Alors : $det_{\mathcal{B}}$ est une forme n-alternée sur E et $det_{\mathcal B}(\mathcal B)=1$ \\
	\textcolor{red}{Démonstration :}
	{\bf n-alternée :} si $\vec x_i= \vec x_k$ avec $i\neq k$ :
	$\forall j \in [[1,n]],a_{ij}=a_{kj}$, $\tau=( i \quad k)$ \\
	$det_{\mathcal B}(\vec x_1,..., \vec x_n) = \sum_{\sigma \in S_n} \epsilon(\sigma) \prod_{j=1}^n a_{\sigma(j),j}$ \\
	$det_{\mathcal B}(\vec x_1,..., \vec x_n) = \sum_{\sigma \in S_n} \epsilon(\sigma) \prod_{j=1}^n   a_{\sigma(j),\tau(j)}$ car $\vec x_i= \vec x_k$ \\
	$det_{\mathcal B}(\vec x_1,..., \vec x_n) = \sum_{\gamma \in S_n} \underbrace{\epsilon(\gamma \tau)}_{=\epsilon(\gamma)\epsilon(\tau)=-\epsilon(\gamma)} \prod_{j=1}^n a_{\gamma(\tau(j)),\tau(j)}$ \\
	$det_{\mathcal B}(\vec x_1,..., \vec x_n) = - \sum_{\gamma \in S_n} \epsilon(\gamma) \prod_{l=1}^n a_{\gamma(l),l}$ en posant $l=\tau(j)$ \\
	$det_{\mathcal B}(\vec x_1,..., \vec x_n) = -det_{\mathcal B}(\vec x_1,...,\vec x_n) $ qui est donc nul \\
	{\boldmath $det_{\mathcal B}(\mathcal B)=1$ :} \\
	$a_{ij}=\delta_{ij}$ \\
		$det_{\mathcal B}(\mathcal B)= \sum_{\sigma \in S_n} \epsilon(\sigma) \prod_{j=1}^n \delta_{\sigma(j),j}$ \\
	si $\sigma\neq id$ : il existe j tel que $\sigma(j)\neq j$ donc $\delta_{\sigma(j),j}=0$ produit nul \\
	$det_{\mathcal B}(\mathcal B)= \epsilon(id) \prod_{j=1}^n \delta_{j,j}$ \\
	$det_{\mathcal B}(\mathcal B)= 1$
	\section{Formule de changement de base des déterminants, caractérisation des bases.}
	\textcolor{green}{Propriété :} \\
	Fixons $\mathcal B$ base de E. Soit $\phi \in A_n(E)$, alors : \\
	$\forall \mathcal F \in E, \phi(\mathcal F)= \phi(\mathcal B)det_{\mathcal B}(\mathcal F)$ \\
	\textcolor{green}{Propriété :} \\
	formule de changement de base. Soient $\mathcal B$ et $\mathcal C$ bases de E. Alors : \\
	$\forall \mathcal F, det_{\mathcal C}(\mathcal F)= det_{\mathcal C}(\mathcal B) det_{\mathcal B}(\mathcal F)$ \\
	\textcolor{red}{Démonstration :} \\
	cas particulier de la propriété précédente avec $\phi= det_{\mathcal C} \in A_n$ \\
	\textcolor{green}{Corrolaire :} \\
	$\mathcal B$ et $\mathcal{C}$ bases de E. Alors : \\
	$det_{\mathcal B}(\mathcal C) \neq 0$ et $det_{\mathcal C}(\mathcal B)= \frac{1}{det_{\mathcal B}(\mathcal C)}$ \\
	\textcolor{red}{Démonstration :} \\
	En prenant $\mathcal F= \mathcal C$ : $\underbrace{det_{\mathcal C}(\mathcal C)}_{=1}=det_{\mathcal C}(\mathcal B)det_{\mathcal B}(\mathcal C)$ \\
	\textcolor{green}{Corrolaire :} \\
	Soit $\mathcal B$ base de E, $\mathcal F \in  E^n$ quelconque . Alors : \\
	$\mathcal F$ base de E $\Longleftrightarrow det_{\mathcal B}(\mathcal F) \neq 0$ \\
	\textcolor{red}{Démonstration :} \\
	$\Rightarrow$ : propriété précédente \\
	$\Leftarrow :$ par contraposition. \\
	Supposons que $\mathcal F$ ne soit pas une base de E. Comme elle a $n=dim(E)$ vecteurs, si elle était libre alors ce serait une base, elle est donc liée. \\
	Commme $det_{\mathcal B} \in A_n(E)$ : \\
	$det_{\mathcal B}(\mathcal F)=0$
	\section{$det_{\mathcal B}(f(\mathcal B))$ ne dépend pas du choix de $\mathcal B$ + $det_{\mathcal B}(f(x_1),...,f(x_n))=det(f)det_{\mathcal B}(x_1,...,x_n) $.}
	\section{Propriété variées de $det(f)$ déduites de la question précédente.}
	\section{$det({}^tA)=det(A)$}
	\section{Déterminant triangulaire}
	\end{document}
