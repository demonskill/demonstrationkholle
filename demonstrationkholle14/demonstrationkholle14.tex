\documentclass{article}
\renewcommand*\familydefault{\sfdefault}
\usepackage[utf8]{inputenc}
\usepackage[T1]{fontenc}
\setlength{\textwidth}{481pt}
\setlength{\textheight}{650pt}
\setlength{\headsep}{10pt}
\usepackage{amsfonts}
\usepackage[T1]{fontenc}
\usepackage{palatino}
\usepackage{calrsfs}
\usepackage{geometry}
\geometry{ left=3cm, top=2cm, bottom=2cm, right=2cm}
\usepackage{xcolor}
\usepackage{amsmath}
\usepackage{tikz,tkz-tab}
\usepackage{cancel}
\usepackage{pgfplots}
\usepackage{pstricks-add}
\usepackage{pst-eucl}
\usepackage{amssymb}
\usepackage{icomma}
\begin{document}
\title{Démonstration kholle 14}
\date{}
\maketitle
	\renewcommand{\thesection}{\Roman{section}}
	\setlength{\parindent}{1.5cm}
\section{Inégalité de Lagrange} 
\textcolor{green}{Propriété :} \\ 

\section{Unicité et produit du D.L}
\textcolor{green}{Propriété :} \\ 
\textcolor{green}{1)} Si on a : \\ 
$f(x)=\sum_{k=0}^n \alpha_k(x-a)^k+o(x^n)$  \\
$f(x)= \sum_{k=0}^n \beta_k(x-a)^k + o(x^n)$ \\ 
Alors : $ \forall k \in [[0,n]], \alpha_k=\beta_k$ \\ 
\textcolor{red}{Démonstration : } \\ 
posons $E= \lbrace k \in [[0,n]]: \alpha_k \neq \beta_k$ \\
si $E \neq \emptyset$ : comme $E\subset \mathbb{N}$, il a un minimum p : \\ 
On a :\\ 
$f(x)= \sum_{k=0}^n \alpha_k (x-a)^k + (x-a)^n \epsilon_1(x)$ où $ \epsilon_1(x) \rightarrow_{x\rightarrow a} 0$ \\ 
$f(x)= \sum_{k=0}^n \beta_k (x-a)^k + (x-a)^n \epsilon_2(x)$ où $\epsilon_2(x) \rightarrow_{x \rightarrow a} 0$ \\ 
ainsi pour tout $x \in d_f$ : \\ 
$\sum_{k=1}^n(\alpha_k- \beta_k) (x-a)^k+ (x-a)^n (\epsilon_1(x) - \epsilon_2(x)) $
\section{Etablir les DL en 0 à tout ordre de $1/(1\pm x)$, puis de $\ln(1+x)$ et $arctan(x)$(primitivation admise)}
\section{$DL_5(0)$ de $\tan(x)$ par produit. Retrouver cela par division selon les puissances croissantes}
\section{$DL_8(0)$ de $\tan(x)$ en exploitant $tan'(x)=1+\tanˆ2(x)$}
\section{$DL_4(0)$ de $\frac{sh(x)-sin(x)}{1-\cos(x)}$}
\end{document}
