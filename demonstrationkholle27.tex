X\documentclass{article}
\renewcommand*\familydefault{\sfdefault}
\usepackage[utf8]{inputenc}
\usepackage[T1]{fontenc}
\setlength{\textwidth}{481pt}
\setlength{\textheight}{650pt}
\setlength{\headsep}{10pt}
\usepackage{amsfonts}
\usepackage[T1]{fontenc}
\usepackage{palatino}
\usepackage{calrsfs}
\usepackage{geometry}
\geometry{ left=3cm, top=2cm, bottom=2cm, right=2cm}
\usepackage{xcolor}
\usepackage{amsmath}
\usepackage{tikz,tkz-tab}
\usepackage{cancel}
\usepackage{pgfplots}
\usepackage{pstricks-add}
\usepackage{pst-eucl}
\usepackage{amssymb}
\usepackage{icomma}
\usepackage{listings}
\begin{document}
\title{Démonstration kholle 27}
\date{}
\maketitle
\renewcommand{\thesection}{\Roman{section}}
\setlength{\parindent}{1.5cm}
\section{Distance d'un point à un sous-espace,atteinte en un unique point (le projeté orthogonal) Exemple : $d(X^2,\mathbb R_1[X])$ pour le produit scalaire défini par $\langle P,Q \rangle= \int_0^1P(t)Q(t)dt$}
\textcolor{green}{Propriété :} \\
Soit F de dimension finie $d(\vec y, F)$ est atteinte en un unique point qui est $p_F(\vec y)$ \\
\textcolor{red}{Démonstration :} \\
si $dim(E)=2$ et $dim(F)=1$ : \\
\setlength{\unitlength}{0.75mm}
\begin{picture}(10,40)
\put(0,20){\vector(4,1){20}}
\put(-2,20){\vector(1,0){40}}
\put(40,20){$F$}
\put(15,14){$P_F(\vec y)$}
\multiput(20,20)(0,1){5}
{\line(0,20){0.5}}
\thicklines
\thinlines
\end{picture}
$\| \vec y - \vec x \| > \| \vec y - p_F(\vec y) \|$ \\
{ \bf Cas général :} $p_F(\vec y)- \vec x \in F$ car $\vec x$ et $p_F(\vec y) \in F$ \\
et $\vec y - p_F(\vec y) \in F^\perp$ car $p_F$ est la projection sur F parrallèle à $F^\perp$ \\
donc par le théorème de Pythagore : \\
$\| \vec y- \vec x \|^2= \| \vec y -P_F(\vec y) \|^2 + \| p_F(\vec y) - \vec x \|^2$ \\
$\|y-x \|^2 \geq \| \vec y-p_F(\vec y) \|^2$ avec égalité si et seulement si $\|p_F(\vec y)- \vec x \|=0$, c'est-à-dire $\vec x=p_F(\vec y)$ \\
{\bf Exemple :} \\
$E=\mathbb R[X]$
$\langle P,Q \rangle = \int_0^1 P(t)Q(t)dt$ \\
Calculer $(d(X^2,\mathbb R_1[X]))^2$ \\
où $\int_0^1(t^2-at-b)^2dt= \| X^2-(aX +b) \|^2$ \\
Il est atteint pour $(a,b)$ tel que $p_F(X^2)=aX+b$ uniquement : \\
$X^2-p_F(X^2) \in (\mathbb R_1[X])^\perp$ \\
donc : \\
$\langle X^2-p_F(X^2),1 \rangle=0$ \\
$\langle X^2-p_F(X^2),X \rangle=0$ \\ \\
$\langle aX+b,1 \rangle = \langle X^2,1 \rangle $ \\
$\langle aX+b,X \rangle = \langle X^2,X \rangle$  \\ \\
$\langle X,1 \rangle a + \langle 1,1 \rangle b= \langle X^2,1 \rangle$ \\
$\langle X,X \rangle a + \langle 1,X \rangle b= \langle X^2,X \rangle$ \\ \\
$\frac{1}{2} a +b = \frac{1}{3}$ \\
$\frac{1}{3}a +\frac{1}{2}b =\frac{1}{4}$ \\ \\
$\begin{pmatrix}
\frac{1}{2} & 1 \\ \\
\frac{1}{3} & \frac{1}{2}
\end{pmatrix}
\begin{pmatrix}
  a \\ \\
  b \\
  \end{pmatrix}
  =
  \begin{pmatrix}
    \frac{1}{3} \\ \\
    \frac{1}{4} \\
    \end{pmatrix}$ \\ \\
$\begin{pmatrix}
a \\
b \\
\end{pmatrix} =
\frac{1}{-1/12}
\begin{pmatrix}
1/2 & -1 \\
-\frac{1}{3} & \frac{1}{2}
\end{pmatrix}
\begin{pmatrix}
\frac 1 3 \\
\frac 1 4
\end{pmatrix}$ \\
$(a,b)=(1, \frac {-1} 6)$ \\
Le minimum cherché est : \\
$\int_0^1 (t^2-t+\frac{1}{4})^2dt$ \\
On peut écrire : \\
$X^2=\underbrace{p_F(X^2)}_{ \in F}+ \underbrace{(X^2-p_F(X^2))}_{\in F^\perp}$ \\
donc par Pythagore : \\
$\| X^2 \|^2= \| p_F(X^2) \|^2+ \underbrace{\|X^2-p_F(X^2)\|^2}_{m}$ \\
$m= \| X^2 \|^2 - \| aX+b \|^2$ \\
$m= \int_0^1 t^4 dt - \int_0^1 (t-\frac{1}{6})^2dt$ \\
ce qui est le plus simple à calculer \\
On obtient : \\
$m= \frac 1 5 - [ \frac 1 2 t^3 - \frac 1 6 t^2 + \frac 1 {36} t]_0^1$ \\
$m= \frac 15 - \frac 1 3 + \frac 1 6 - \frac 1 {36}= \frac 1 {180}$
\section{Si $E$ est un espace euclidien, isomorphisme \\ $\phi : u \in E \mapsto \langle u, \cdot \rangle \in \mathcal L(E, \mathbb R)$}
\textcolor{green}{Théorème :} \\
Soit E euclidien et $n=dim(E) \in \mathbb N^*$, l'application : \\
$\phi : E \rightarrow \mathcal L (E, \mathbb R)$ \\
$\vec u \mapsto \langle \vec u, \cdot \rangle $ \\
est un isomorphisme \\
\textcolor{red}{Démonstration :} \\
Montrons que {\bf \boldmath $\phi$ est linéaire} c'est-à-dire : $\phi \in \mathcal L(E,\mathcal L(E,\mathbb R))$ \\
Soit $(\vec u, \vec v) \in E^2$ et $\lambda \in \mathbb R$ \\
Pour $\vec x \in E :$ \\
$\langle \vec u + \lambda \vec vn \vec x \rangle = \langle \vec u ,\vec x \rangle + \lambda \langle \vec v, \vec u \rangle$ \\
nombre réels  donc : $\phi(\vec u+ \lambda \vec v)(\vec x)=\phi(\vec u)(\vec x) + \lambda \phi(\vec v)(\vec x)$ \\
Vrai pour tout $\vec x$ donc : \\
application de E dans $\mathbb R$ $\phi(\vec u + \lambda \vec v)=\phi(\vec u)+ \lambda \phi(\vec v)$ \\
{\bf \boldmath Soit $\vec u \in Ker(\phi)$} : \\
$\phi(\vec u)= \tilde{0}$ \\
$\forall \vec x \in E, \underbrace{\phi(\vec u, \vec x)}_{\langle \vec u,\vec x \rangle}=0$ \\
pour $\vec x = \vec u$ : $\| \vec u \|^2=0$ donc $\vec u= \vec 0$ \\
$Ker(\phi)= \lbrace \vec 0 \rbrace$ \\
{\bf Théorème du rang :} \\
$dim(Im(\phi))= dim(E)-dim(Ker(\phi))$ \\
$dim(Im(\phi))=n=dim(\mathcal L(E,\mathbb R))$ \\
or $Im(\phi) \subset \mathcal L(E,\mathbb R)$ donc $Im(\phi) = \mathcal L(E, \mathbb R)$
\section{Isométries vectorielles : $O(E)$ est un sous-groupe de $GL(E)$, équivalence entre conservation de la norme et conservation du produit scalaire, caractérisation par l'image d'une base orthonormée}
\textcolor{green}{Propriété :} \\
l'ensemble $O(E)$ des isometries vectorielles d'un espace euclidien E est un sous-groupe de $(GL(E),\circ)$. C'est le groupe orthogonal de E, les isometries sont aussi appelée automorphismes orthogonaux \\
\textcolor{red}{Démonstration :} \\
Montrons que { \boldmath $O(E) \subset GL(E) :$} \\
Soit $f \in O(E) : f \in \mathcal L(E)$ et : \\
$\forall \vec x \in E, \| f(\vec x) \|= \|\vec x \|$ \\
Soit $\vec x \in Ker(f) : \|\vec x \|= \|f(\vec x)\| = \| \vec 0 \| =0$ donc $\vec x =\vec 0$ \\
f est un endomorphisme injectif d'un espace de dimension finie don f est bijectif (corrolaire du théorème du rang) \\
{\boldmath $O(E) \neq \emptyset$}  car $Id_E \in O(E)$ \\
Soit {\boldmath $(f,g) \in (O(E))^2} :$ }\\
Pour $\vec x \in E$ : \\
$\|f(g(\vec x))\| =\| g(\vec x) \| = \|\vec x \|$ car  $(f,g) \in (O(E))^2$ \\
donc $fg \in O(E)$ \\
Soit $f \in O(E)$ : \\
Soit $\vec x \in E$ \\
avec $\vec y=f^-1(\vec x) \in E :$ \\
$\|f(\vec y) \|= \| \vec y \|$ car $f \in O(E)$ \\
$\|\vec x \| = \|f^{-1}(\vec x) \|$ ainsi {\boldmath $f^{-1} \in O(E)$ } \\
\textcolor{green}{Propriété :} soit $f \in \mathcal L(E)$ Les proposition suivante sont équivalentes : \\
\textcolor{green}{1)} $\forall \vec x \in E, \| f(\vec x) \| = \| \vec x \|$ \\
\textcolor{green}{2)} $\forall (\vec x, \vec y) \in E^2,\langle f(\vec x), f(\vec y) \rangle =\langle \vec x, \vec y$ \\
\textcolor{red}{Démonstration :} \\
\textcolor{green}{2} $\Rightarrow$ \textcolor{green}{1} ; Soit $\vec x \in E :$ Avec $\vec y=\vec x :$ \\
$\| f(\vec x) \|^2=\|\vec x \|^2$ \\
$\| f(\vec x) \|=\| \vec x \|$ car les normes sont positives \\
\textcolor{green}{1} $\Rightarrow$ \textcolor{green}{2} : Soit $(\vec x, \vec y) \in E^2$ \\
polarisation : \\
$2 \langle f(\vec x), f(\vec y) \rangle = \| f(\vec x) + f(\vec y) \|^2- \| f(\vec x) \|^2 - \|f(\vec y) \|^2$ \\
comme $f\in \mathcal L(E)$, $2 \langle f(\vec x), f(\vec y) \rangle =\| f(\vec x + \vec y) \|^2- \| f(\vec x) \|^2 - \|f(\vec y) \|^2$ \\
Par hypothèse : $2\langle f(\vec x),f(\vec y) \rangle = \| \vec x+ \vec y \|^2- \| \vec x \|^2 - \|\vec y \|^2$ \\
par polarisation: $\langle f(\vec x),f(\vec y) \rangle = \langle \vec x, \vec y \rangle$ \\
\textcolor{green}{Propriété :} E euclidien, $f\in \mathcal L(E)$ Les propriété suivante sont équivalentes : \\
\textcolor{green}{1)} $f\in O(E)$ \\
\textcolor{green}{2)} Pour toute base orthonormée $\mathcal B$ de E $f(\mathcal B)$ est une base orthonormée de E \\
\textcolor{green}{3)} Il existe une base orthonormée $\mathcal B$ de $E$ tel que $f(\mathcal B)$ soient une base orthonormée de E \\
\textcolor{red}{Démonstration :} \\
notons $n=dim(E) \in \mathbb N^*$ \\
\textcolor{green}{1} $\Rightarrow$ \textcolor{green}{2} $f \in O(E)$ \\
Soit $\mathcal B=(\vec e_1,...,\vec e_n)$ une base orthonormée de E \\
Alors pour $(i,j) \in [[1,n]]^2 :$ \\
$\langle f(\vec e_i), f(\vec e_j) \rangle=\langle \vec e_i,\vec e_j \rangle$ car $f\in O(E)$ \\
$\langle f(\vec e_i), f(\vec e_j) \rangle=\delta_{ij}$ car $\mathcal B$ orthonormée. \\
ainsi: $f(\mathcal{B})$ est une famille orthonormée donc libre. \\
Comme elle a $n=dim(E)$ vecteurs, c'est une base orthonormée de E \\
\textcolor{green}{2} $\Rightarrow$ \textcolor{green}{3} : Il existe au moins une base orthonormée de E elle convient (par \textcolor{green}{2}) \\
\textcolor{green}{3} $\Rightarrow$ \textcolor{green}{1} : Soit $\mathcal B$ une base orthonormée de E tel que $f (\mathcal B)$ le soit aussi on a donc: \\
$\forall(i,j) \in [[1,n]]^2, \langle f(\vec e_i),f(\vec e_j) \rangle = \langle \vec e_i, \vec e_j \rangle =\delta_{ij}$ \\
Soit $(\vec x, \vec y) \in E^2$ \\
On a : $\vec x= \sum_{i=1}^n \langle \vec x, \vec e_i \rangle \vec e_i$ \\
$\vec y= \sum_{i=1}^n \langle \vec y, \vec e_i \rangle  \vec e_i$ \\
donc comme $\mathcal B$ est une base orthonormée donc : \\
$\langle \vec x, \vec y \rangle= \sum_{i=1}^n \langle \vec x, \vec e_i \rangle \langle \vec y, \vec e_i \rangle$ \\
en appliquant f à $\vec x$ et $\vec y$ : \\
$f(\vec x)=\sum_{i=1}^n \langle \vec x, \vec e_i \rangle f(\vec e_i)$ car $f\in \mathcal L(E)$ \\
$f(\vec y)= \sum_{i=1}^n \langle \vec y, \vec e_i \rangle f(\vec e_i)$ \\
Or $f(\mathcal B)$ est une base orthonormée de E : \\
$\langle f(\vec x), f(\vec y) \rangle = \sum_{i=1}^n \langle \vec x, \vec e_i \rangle \langle \vec y, \vec e_i \rangle$ \\
$\langle f(\vec x), f(\vec y) \rangle =\langle \vec x,\vec y \rangle $
\section{Démontrer que $O_n(\mathbb R)$ est un sous-groupe de $GL_n(\mathbb R)$ + description de $O_2 (\mathbb R)$ + commutativité de $SO_2(\mathbb R)$}
\textcolor{green}{Propriété :} \\
$O_n(\mathbb R)$ est un sous-groupe de $(Gl_n(\mathbb R),X\times )$ \\
\textcolor{red}{Démonstration :} \\
{\boldmath $O_n(\mathbb R) \subset GK_n(\mathbb R)$} par définition \\
{\boldmath $O_n(\mathbb R) \neq \emptyset$} car $I_n \in O_n(\mathbb R)$ \\
Soit {\boldmath $(A,B) \in (O_n(\mathbb R))^2$} : \\
$(AB)^{-1}=B^{-1}A^{-1}={}^tB{}^tA={}^t(AB)$ \\
\textcolor{green}{Propriété :} soit $A\in O_2(\mathbb R)$ : \\
- si $det(A)=1$ : \\
$A=\begin{pmatrix}
\cos(\theta) & -\sin(\theta) \\
\sin(\theta) & \cos(\theta)
\end{pmatrix}$
($\theta \in \mathbb R$ unique modulo $2\pi$)  et toute matrice de cette forme de $O_2^+(\mathbb R)$ \\
- si $det(A)=-1$ : \\
$A=\begin{pmatrix}
\cos(\theta) & \sin(\theta) \\
\sin(\theta) & -\cos(\theta)
\end{pmatrix}$ \\
($\theta \in \mathbb R$ unique modulo $2 \pi$) et tout matrice de cette forme $\in O_2^-(\mathbb R)$ \\
\textcolor{red}{Démonstration :} \\
Il est clair que toute matrices de ces formes conviennent \\
Réciproquement, soit $A\in O_1(\mathbb R) :$ \\
$A=\begin{pmatrix}
a & c \\
b & d \\
\end{pmatrix}$
colonnes orthonormés donc : \\
$a^2+b^2=1$\\
$c^2 +d^2=1$ \\
$ac+bd=0$ \\
Il existe $(\theta,\phi) \in \mathbb R^2$ tel que: \\
$a=\cos(\theta),b=\sin(\theta)$ \\
$c=\cos(\phi),d=\sin(\phi)$ \\
or : $0= ac+bd=  \cos(\phi-\theta)$ \\
donc $\phi- \theta \equiv \pm \frac \pi 2 [2\pi]$ \\
Si $\phi - \theta \equiv \frac \pi 2 [2 \pi ]$ : \\
$c=\cos(\phi)=\cos(\theta + \frac \pi 2)=-\sin(\theta)$ \\
$d=\sin(\phi)=\sin(\theta + \frac \pi 2)=\cos(\theta)$ \\
$A=\begin{pmatrix}
\cos(\theta) &- \sin(\theta) \\
\sin(\theta) & \cos(\theta)
\end{pmatrix}$ du premier type  \\
si $\phi - \theta \equiv - \frac \pi 2 [2\pi] :$ \\
$c=\cos(\phi)=\cos(\theta-\frac \pi 2)= \sin(\theta)$ \\
$d=\sin(\phi)=\sin(\theta-\frac \pi 2)=-\cos(\theta)$ \\
$A=\begin{pmatrix}
\cos(\theta) & \sin(\theta) \\
\sin(\theta) & -\cos(\theta)
\end{pmatrix}$ du deuxième type \\
\textcolor{green}{Propriété :} \\
$SO_2(\mathbb R)$ est commutatif\\
\textcolor{red}{Démonstration :} \\
$\begin{pmatrix}
\cos(\theta) & -\sin(\theta) \\
\sin(\theta) & \cos(\theta)
\end{pmatrix}
\begin{pmatrix}
\cos(\phi) & - \sin(\phi) \\
\sin(\phi) & \cos(\phi)
\end{pmatrix}
=
\begin{pmatrix}
\cos(\theta + \phi) & - \sin(\theta + \phi) \\
\sin(\theta + \phi) & \cos(\theta + \phi)
\end{pmatrix}$
\section{Etude de $SO(E)$ ($E$ plan vectoriel orienté)}
\textcolor{green}{Propriété :} \\
E: plan euclidien orientée \\
$f \in SO(E)$ \\
\textcolor{green}{1)}
\section{Etude de $O^-(E)$($E$ plan vectoriel orienté)}
\end{document}
