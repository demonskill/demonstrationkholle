X\documentclass{article}
\renewcommand*\familydefault{\sfdefault}
\usepackage[utf8]{inputenc}
\usepackage[T1]{fontenc}
\setlength{\textwidth}{481pt}
\setlength{\textheight}{650pt}
\setlength{\headsep}{10pt}
\usepackage{amsfonts}
\usepackage[T1]{fontenc}
\usepackage{palatino}
\usepackage{calrsfs}
\usepackage{geometry}
\geometry{ left=3cm, top=2cm, bottom=2cm, right=2cm}
\usepackage{xcolor}
\usepackage{amsmath}
\usepackage{tikz,tkz-tab}
\usepackage{cancel}
\usepackage{pgfplots}
\usepackage{pstricks-add}
\usepackage{pst-eucl}
\usepackage{amssymb}
\usepackage{icomma}
\usepackage{listings}
\begin{document}
\title{Démonstration kholle 27}
\date{}
\maketitle
\renewcommand{\thesection}{\Roman{section}}
\setlength{\parindent}{1.5cm}
\section{Distance d'un point à un sous-espace,atteinte en un unique point (le projeté orthogonal) Exemple : $d(X^2,\mathbb R_1[X])$ pour le produit scalaire défini par $\langle P,Q \rangle= \int_0^1P(t)Q(t)dt$}
\textcolor{green}{Propriété :} \\
Soit F de dimension finie $d(\vec y, F)$ est atteinte en un unique point qui est $p_F(\vec y)$ \\
\textcolor{red}{Démonstration :} \\
si $dim(E)=2$ et $dim(F)=1$ : \\
\setlength{\unitlength}{0.75mm}
\begin{picture}(10,40)
\put(0,20){\vector(4,1){20}}
\put(-2,20){\vector(1,0){40}}
\put(40,20){$F$}
\put(15,14){$P_F(\vec y)$}
\multiput(20,20)(0,1){5}
{\line(0,20){0.5}}
\thicklines
\thinlines
\end{picture}
$\| \vec y - \vec x \| > \| \vec y - p_F(\vec y) \|$ \\
{ \bf Cas général :} $p_F(\vec y)- \vec x \in F$ car $\vec x$ et $p_F(\vec y) \in F$ \\
et $\vec y - p_F(\vec y) \in F^\perp$ car $p_F$ est la projection sur F parrallèle à $F^\perp$ \\
donc par le théorème de Pythagore : \\
$\| \vec y- \vec x \|^2= \| \vec y -P_F(\vec y) \|^2 + \| p_F(\vec y) - \vec x \|^2$ \\
$\|y-x \|^2 \geq \| \vec y-p_F(\vec y) \|^2$ avec égalité si et seulement si $\|p_F(\vec y)- \vec x \|=0$, c'est-à-dire $\vec x=p_F(\vec y)$ \\
{\bf Exemple :} \\
$E=\mathbb R[X]$
$\langle P,Q \rangle = \int_0^1 P(t)Q(t)dt$ \\
Calculer $(d(X^2,\mathbb R_1[X]))^2$ \\
où $\int_0^1(t^2-at-b)^2dt= \| X^2-(aX +b) \|^2$ \\
Il est atteint pour $(a,b)$ tel que $p_F(X^2)=aX+b$ uniquement : \\
$X^2-p_F(X^2) \in (\mathbb R_1[X])^\perp$ \\
donc : \\
$\langle X^2-p_F(X^2),1 \rangle=0$ \\
$\langle X^2-p_F(X^2),X \rangle=0$ \\ \\
$\langle aX+b,1 \rangle = \langle X^2,1 \rangle $ \\
$\langle aX+b,X \rangle = \langle X^2,X \rangle$  \\ \\
$\langle X,1 \rangle a + \langle 1,1 \rangle b= \langle X^2,1 \rangle$ \\
$\langle X,X \rangle a + \langle 1,X \rangle b= \langle X^2,X \rangle$ \\ \\
$\frac{1}{2} a +b = \frac{1}{3}$ \\
$\frac{1}{3}a +\frac{1}{2}b =\frac{1}{4}$ \\ \\
$\begin{pmatrix}
\frac{1}{2} & 1 \\ \\
\frac{1}{3} & \frac{1}{2}
\end{pmatrix}
\begin{pmatrix}
  a \\ \\
  b \\
  \end{pmatrix}
  =
  \begin{pmatrix}
    \frac{1}{3} \\ \\
    \frac{1}{4} \\
    \end{pmatrix}$ \\ \\
$\begin{pmatrix}
a \\
b \\
\end{pmatrix} =
\frac{1}{-1/12}
\begin{pmatrix}
1/2 & -1 \\
-\frac{1}{3} & \frac{1}{2}
\end{pmatrix}
\begin{pmatrix}
\frac 1 3 \\
\frac 1 4
\end{pmatrix}$ \\
$(a,b)=(1, \frac {-1} 6)$ \\
Le minimum cherché est : \\
$\int_0^1 (t^2-t+\frac{1}{4})^2dt$ \\
On peut écrire : \\
$X^2=\underbrace{p_F(X^2)}_{ \in F}+ \underbrace{(X^2-p_F(X^2))}_{\in F^\perp}$ \\
donc par Pythagore : \\
$\| X^2 \|^2= \| p_F(X^2) \|^2+ \underbrace{\|X^2-p_F(X^2)\|^2}_{m}$ \\
$m= \| X^2 \|^2 - \| aX+b \|^2$ \\
$m= \int_0^1 t^4 dt - \int_0^1 (t-\frac{1}{6})^2dt$ \\
ce qui est le plus simple à calculer \\
On obtient : \\
$m= \frac 1 5 - [ \frac 1 2 t^3 - \frac 1 6 t^2 + \frac 1 {36} t]_0^1$ \\
$m= \frac 15 - \frac 1 3 + \frac 1 6 - \frac 1 {36}= \frac 1 {180}$
\section{Si $E$ est un espace euclidien, isomorphisme \\ $\phi : u \in E \mapsto \langle u, \cdot \rangle \in \mathcal L(E, \mathbb R)$}
\textcolor{green}{Théorème :} \\
Soit E euclidien et $n=dim(E) \in \mathbb N^*$, l'application : \\
$\phi : E \rightarrow \mathcal L (E, \mathbb R)$ \\
$\vec u \mapsto \langle \vec u, \cdot \rangle $ \\
est un isomorphisme \\
\textcolor{red}{Démonstration :} \\
Montrons que {\bf \boldmath $\phi$ est linéaire} c'est-à-dire : $\phi \in \mathcal L(E,\mathcal L(E,\mathbb R))$ \\
Soit $(\vec u, \vec v) \in E^2$ et $\lambda \in \mathbb R$ \\
Pour $\vec x \in E :$ \\
$\langle \vec u + \lambda \vec vn \vec x \rangle = \langle \vec u ,\vec x \rangle + \lambda \langle \vec v, \vec u \rangle$ \\
nombre réels  donc : $\phi(\vec u+ \lambda \vec v)(\vec x)=\phi(\vec u)(\vec x) + \lambda \phi(\vec v)(\vec x)$ \\
Vrai pour tout $\vec x$ donc : \\
application de E dans $\mathbb R$ $\phi(\vec u + \lambda \vec v)=\phi(\vec u)+ \lambda \phi(\vec v)$ \\
{\bf \boldmath Soit $\vec u \in Ker(\phi)$} : \\
$\phi(\vec u)= \tilde{0}$ \\
$\forall \vec x \in E, \underbrace{\phi(\vec u, \vec x)}_{\langle \vec u,\vec x \rangle}=0$ \\
pour $\vec x = \vec u$ : $\| \vec u \|^2=0$ donc $\vec u= \vec 0$ \\
$Ker(\phi)= \lbrace \vec 0 \rbrace$ \\
{\bf Théorème du rang :} \\
$dim(Im(\phi))= dim(E)-dim(Ker(\phi))$ \\
$dim(Im(\phi))=n=dim(\mathcal L(E,\mathbb R))$ \\
or $Im(\phi) \subset \mathcal L(E,\mathbb R)$ donc $Im(\phi) = \mathcal L(E, \mathbb R)$
\section{Isométries vectorielles : $O(E)$ est un sous-groupe de $GL(E)$, équivalence entre conservation de la norme et conservation du produit scalaire, caractérisation par l'image d'une base orthonormée}
\section{Démontrer que $O_n(\mathbb R)$ est un sous-groupe de $GL_n(\mathbb R)$ + description de $O_2 (\mathbb R)$ + commutativité de $SO_2(\mathbb R)$}
\section{Etude de $SO(E)$ ($E$ plan vectoriel orienté)}
\section{Etude de $O^-(E)$($E$ plan vectoriel orienté)}
\end{document}
