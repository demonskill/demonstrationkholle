\documentclass{article}
\renewcommand*\familydefault{\sfdefault}
\usepackage[utf8]{inputenc}
\usepackage[T1]{fontenc}
\setlength{\textwidth}{481pt}
\setlength{\textheight}{650pt}
\setlength{\headsep}{10pt}
\usepackage{amsfonts}
\usepackage[T1]{fontenc}
\usepackage{palatino}
\usepackage{calrsfs}
\usepackage{geometry}
\geometry{ left=3cm, top=2cm, bottom=2cm, right=2cm}
\usepackage{xcolor}
\usepackage{amsmath}
\usepackage{tikz,tkz-tab}
\usepackage{cancel}
\usepackage{pgfplots}
\usepackage{pstricks-add}
\usepackage{pst-eucl}
\usepackage{amssymb}
\usepackage{icomma}
\usepackage{listings}
\begin{document}
\title{Démonstration kholle 27}
\date{}
\maketitle
\renewcommand{\thesection}{\Roman{section}}
\setlength{\parindent}{1.5cm}
\section{Distance d'un point à un sous-espace,atteinte en un unique point (le projeté orthogonal) Exemple : $d(X^2,\mathbb R_1[X])$ pour le produit scalaire défini par $\langle P,Q \rangle= \int_0^1P(t)Q(t)dt$}
\section{Si $E$ est un espace euclidien, isomorphisme \\ $\phi : u \in E \mapsto \langle u, \cdot \rangle \in \mathcal L(E, \mathbb R)$}
\section{Isométries vectorielles : $O(E)$ est un sous-groupe de $GL(E)$, équivalence entre conservation de la norme et conservation du produit scalaire, caractérisation par l'image d'une base orthonormée}
\section{Démontrer que $O_n(\mathbb R)$ est un sous-groupe de $GL_n(\mathbb R)$ + description de $O_2 (\mathbb R)$ + commutativité de $SO_2(\mathbb R)$}
\section{Etude de $SO(E)$ ($E$ plan vectoriel orienté)}
\section{Etude de $O^-(E)$($E$ plan vectoriel orienté)}
\end{document}
