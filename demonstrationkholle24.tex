\documentclass{article}
\renewcommand*\familydefault{\sfdefault}
\usepackage[utf8]{inputenc}
\usepackage[T1]{fontenc}
\setlength{\textwidth}{481pt}
\setlength{\textheight}{650pt}
\setlength{\headsep}{10pt}
\usepackage{amsfonts}
\usepackage[T1]{fontenc}
\usepackage{palatino}
\usepackage{calrsfs}
\usepackage{geometry}
\geometry{ left=3cm, top=2cm, bottom=2cm, right=2cm}
\usepackage{xcolor}
\usepackage{amsmath}
\usepackage{tikz,tkz-tab}
\usepackage{cancel}
\usepackage{pgfplots}
\usepackage{pstricks-add}
\usepackage{pst-eucl}
\usepackage{amssymb}
\usepackage{icomma}
\usepackage{listings}
\begin{document}
\title{Démonstration kholle 24}
\date{}
\maketitle
	\renewcommand{\thesection}{\Roman{section}}
	\setlength{\parindent}{1.5cm}
        \section{Matrice de $f(\vec x)$, de $g \circ f $}
        \section{Définition des matrices de passage et démonstration des formules de changement de base (vecteurs, applications linéaires en génénral, endomorphisme)}
        \section{Relation de similitude : c'est une relation d'équivalence sur $M_n( \mathbb K)$, invariance de la trace, trace d'un projecteur.}
        \section{si $f \in \mathcal L(E,F)$ est de rang r, il existe $\mathcal U$ base de $E$ et $\mathcal V$ base de $F$ telles que $Mat_{$\mathcal{UV}$}(f)=J_{n,p,r}$}
        \section{Si $f \in mathcal L(E,F)$ et s'il existe $\mathcal U$ base de $E$ et $\mathcal V$ base de $F$ telles que $Mat_{$\mathcal{UV}$}(f)=J_{n,p,r}$, $rg(f)=r$}
        \section{Une matrice $n \times p$ est de rang $r$ si et seulemnt si, elle est équivalente à$J_{n,p,r}$. Application au rang de la transposée}
\end{document}
