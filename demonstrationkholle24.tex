\documentclass{article}
\renewcommand*\familydefault{\sfdefault}
\usepackage[utf8]{inputenc}
\usepackage[T1]{fontenc}
\setlength{\textwidth}{481pt}
\setlength{\textheight}{650pt}
\setlength{\headsep}{10pt}
\usepackage{amsfonts}
\usepackage[T1]{fontenc}
\usepackage{palatino}
\usepackage{calrsfs}
\usepackage{geometry}
\geometry{ left=3cm, top=2cm, bottom=2cm, right=2cm}
\usepackage{xcolor}
\usepackage{amsmath}
\usepackage{tikz,tkz-tab}
\usepackage{cancel}
\usepackage{pgfplots}
\usepackage{pstricks-add}
\usepackage{pst-eucl}
\usepackage{amssymb}
\usepackage{icomma}
\usepackage{listings}
\begin{document}
\title{Démonstration kholle 24}
\date{}
\maketitle
\renewcommand{\thesection}{\Roman{section}}
	\setlength{\parindent}{1.5cm}
        \section{Matrice de $f(\vec x)$, de $g \circ f $}
        \textcolor{green}{Propriété :} \\
        E: $\mathbb K$ espace-vectoriel de dimension finie $p\geq 1$ \\
        $\mathcal B=(\vec u_1,...,\vec u_p)$ base de E \\
        F: $\mathbb K$ espace-vectoriel de dimension finie $n\geq 1$ \\
        $\mathcal C=(\vec v_1,...,\vec v_n)$ base de F \\
        $f \in \mathcal L (E,F)$, $\vec x \in E$ \\
        $\underbrace{Mat_{\mathcal{C}}(f(\vec x))}_{n\times 1} = \underbrace{Mat(f)_{\mathcal{BC}}}_{n\times p} \underbrace{Mat(\vec x)_{\mathcal B}}_{p \times 1}$ \\
        \textcolor{red}{Démonstration :} \\
        Notons : $Mat(\vec x)_{\mathcal B}= \begin{pmatrix} x_1 \\ . \\ .  \\ . \\ x_p \end{pmatrix}$ \\
        c'est-à-dire : $\vec x = \sum_{k=1}^p x_k \vec u_k$  \\
        et $Mat (f)_{\mathcal{BC}} = (a_{ij})$ $Mat (f(\vec x))_{\mathcal C} =\begin{pmatrix} y_1 \\ . \\ . \\ . \\ y_n \end{pmatrix}$ \\
        c'est-à-dire : $f(\vec x) = \sum_{i=1}^p y_i \vec v_i$ \\
        Or : $f(\vec x)= \sum_{k=1}^p x_k f(\vec u_k)$ car $f \in \mathcal L (E,F)$ \\
        $f(\vec x)= \sum_{k=1}^p x_k (\sum_{i=1}^n a_{ik} \vec v_i)$ par définition de $Mat(f)_{\mathcal{BC}}$  \\
        $f(\vec x)= \sum_{k=1}^p \sum_{i=1}^n x_k a_{ik} \vec v_i$ \\
        $f(\vec x)= \sum_{i=1}^n (\sum_{k=1}^p a_k x_k) \vec v_i$ \\
        on a $(\sum_{k=1}^p a_k x_k)=y_i$ donc le coefficient de i de $Mat(f)_{\mathcal{BC}} Mat(\vec x)_{\mathcal B}=y_i$ \\
        \textcolor{green}{Propriété :} $E: \mathbb K$ espace vectoriel de dimension finie $r \geq 1$ \\
        $F: \mathbb K$ espace vectoriel de dimension finie $p \geq 1$ \\
        $G: \mathbb K$ espace vectoriel de dimension finie $n \geq 1$ \\
        $f \in \mathcal L(E,F), g \in \mathcal L(F,G)$ \\
        $\mathcal B =( \vec u_1,..., \vec u_r) $ base de E \\
        $\mathcal C =( \vec v_1,..., \vec v_p) $ base de F \\
        $\mathcal D =( \vec w_1,..., \vec w_n) $ base de G \\
        Alors : \\
        $\underbrace{Mat(g \circ f)_{\mathcal{BD}}}_{n \times r}=\underbrace{Mat(g)_{\mathcal{CD}}}_{n \times p} \underbrace{Mat(f)_{\mathcal{BC}}}_{p \times r}$ \\
        \textcolor{red}{Démonstration :} \\
        Notons : $Mat(g)_{\mathcal{CD}}=(a_{ij})$
        $Mat(f)_{\mathcal{BC}}=(b_{ij})$
        $Mat(g \circ f)_{\mathcal{BD}}=(c_{ij})$  \\
        Par définition : \\
        $\forall j \in [[1,r]], f(\vec u_j)=\sum_{k=1}^p b_{kj} \vec v_k$\\
        et :$\forall k \in [[1,p]], g(\vec v_k)=\sum_{i=1}^n a_{ik} \vec w_i$ \\
        Alors pour $j \in [[ 1,r ]] $ : \\
        $g(f(\vec u_j))= \sum_{k=1}^p b_{kj} g(\vec v_k)$ car g linéaire \\
        $g(f(\vec u_j))=\sum_{k=1}^p(b_{kj}\sum_{i=1}^n a_{ik} \vec w_i)$ \\
        $g(f(\vec u_j))=\sum_{i=1}^n\underbrace{(\sum_{k=1}^p a_{ik} b_{kj})}_{=c_{ij}} \vec w_i$ \\
        \section{Définition des matrices de passage et démonstration des formules de changement de base (vecteurs, applications linéaires en général, endomorphisme)}
				\textcolor{red}{Définition :} \\
				On  note la matrice de passage de $\mathcal B$ à $\mathcal C$ $P_{\mathcal{BC}}= Mat(\mathcal C)_{\mathcal B}= Mat(Id_E)_{\mathcal{CB}}$ \\
				\textcolor{green}{Propriété :} Pour $\vec x \in E$ : \\
				$Mat(\vec x)_{\mathcal B}=P_{\mathcal{BC}} Mat_{\mathcal C}(\vec x)$ \\
				\textcolor{red}{Démonstration :}  \\
				$P_{\mathcal{BC}}Mat_{\mathcal C}(\vec x)= Mat(Id_E)_{\mathcal{CB}} Mat(\vec x)_{\mathcal C}$ \\
				\textcolor{green}{Propriété :} \\
				$E: \mathbb K$ espace vectoriel de dimension $p \in \mathbb N^*$ \\
				$F: \mathbb K$ espace vectoriel de dimension $n \in \mathbb N^*$ \\
				$\mathcal B$ et $\mathcal C$, bases de  E \\
				$\mathcal u$ et $\mathcal v$, bases de F alors : \\
				$Mat(f)_{\mathcal{CV}}= \underbrace{P_{\mathcal{vu}}}_{n \times n} \underbrace{Mat(f)_{\mathcal{Bu}}}_{n \times p}  \underbrace{P_{\mathcal{BC}}}_{p \times p}$ \\
				\textcolor{red}{Démonstration :} \\
				$P_{\mathcal{vu}} Mat(f)_{\mathcal{Bu}} P_{\mathcal{BC}}=Mat(Id_F)_{\mathcal{uv}} Mat(f)_{\mathcal{Bu}} Mat(Id_E)_{\mathcal{CB}}$ \\
				$P_{\mathcal{vu}} Mat(f)_{\mathcal{Bu}} P_{\mathcal{BC}} = Mat(Id_f \circ f \circ Id_E)_{\mathcal{Cv}}$ \\
				$P_{\mathcal{vu}} Mat(f)_{\mathcal{Bu}}  P_{\mathcal{BC}} = Mat(f)_{\mathcal{Cv}}$ \\
				\textcolor{green}{Corrolaire} cas d'un endomorphisme \\
				$dim(E)=n \in \mathbb N^*$ $(p=n)$ \\
				$\mathcal {B,C}$ bases de E $f \in \mathcal L (E)$  alors : \\
				$Mat(f)_{\mathcal C}=P_{\mathcal{CB}} Mat(f)_{\mathcal{Bu}} P_{\mathcal{BC}}$ \\
				$Mat(f)_{\mathcal C}=P^{-1}_{\mathcal{BC}} Mat(f)_{\mathcal{Bu}} P_{\mathcal{BC}}$

        \section{Relation de similitude : c'est une relation d'équivalence sur $M_n( \mathbb K)$, invariance de la trace, trace d'un projecteur.}
				\textcolor{green}{Propriété :} \\
				la similitude est une relation d'équivalence. \\
				\textcolor{red}{Démonstration :} \\
				{\bf reflexivité :} Soit $A \in M_n(\mathbb K)$ avec $P=I_n \in Gl_n(\mathbb K)$ : \\
				$A=P^{-1}A P$ \\
				{\bf symétrie :} s'il existe $P \in Gl_n(\mathbb K)$ tel que $B=P^{-1}A P$ : \\
				 $A=Q^{-1}B Q$ avec $Q=P^{-1} \in Gl_n(\mathbb K)$ \\
				 {\bf transitivité :} \\
				 si $B=P^{-1}AP$ et $C=Q^{-1}BQ$ avec $(B,Q) \in GL_n(\mathbb K)$ \\
				 alors $C=R^{-1}AR$ où $R=PQ \in Gl_n(\mathbb K)$ \\
				 \textcolor{green}{Propriété :} \\
				  si A et B sont semblables : $tr(A)=tr(B)$ \\
				 \textcolor{red}{Démonstration :}
        \section{si $f \in \mathcal L(E,F)$ est de rang r, il existe $\mathcal U$ base de $E$ et $\mathcal V$ base de $F$ telles que $Mat_{\mathcal{UV}}(f)=J_{n,p,r}$}
        \section{Si $f \in \mathcal L(E,F)$ et s'il existe $\mathcal U$ base de $E$ et $\mathcal V$ base de $F$ telles que $Mat_{\mathcal{UV}}(f)=J_{n,p,r}$, $rg(f)=r$}
        \section{Une matrice $n \times p$ est de rang $r$ si et seulemnt si, elle est équivalente à$J_{n,p,r}$. Application au rang de la transposée}
\end{document}
